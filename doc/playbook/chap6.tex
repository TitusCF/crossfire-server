
\chapter{Equipment}\label{chap:equip}

\section{Going to market..}\index{equipment, value}
\inputimage{shops}

You can find equipment for sale at easily recognizable buildings. To buy an
item just pick it up and walk out of the building by stepping on a
shop mat\inputimage{shopmat}. The cost of the item will auto-matically be
deducted from your money\inputimage{money}. To sell an item,
enter the shop and drop the item
on the shop floor. Money from the sale will auto-matically be placed in your
inventory. Use the {\tt examine} command, or the cursor and left button
of the mouse to examine the price of an item {\em before} you buy or sell.

\subsection{Some notes about shopping}
\index{shopping}

Most items will have a value based on their ``standard'' cost
multiplied by a factor based on your charisma (see table \ref{tab:pri_eff}).
You can never look good enough that you can buy stuff then sell
it at a profit. \\

\noindent{Some} notable exceptions to the above:
\begin{quote}
   $\bullet$ Gems\index{gems}\inputimage{gems} will always be sold or bought 3\% more or less their standard value. \\
   $\bullet$ For magic stuff value is 3$\times$({\tt magic})$^{3}$ of standard value. \\
   $\bullet$ Unidentified items value is 2/3 of standard.
\end{quote}

\subsection{Plundering shops}
\index{stealing from shops}

It is not possible to steal from shops (sorry!). If you somehow make
it out of a shop with ``unpaid'' items, you will find that they will
be unusable until paid for. On another note, if you save yourself with
unpaid items in a shop, then crash the game and reload, you will
find that the unpaid items will not be saved.

\section{Items}
\label{sec:items}

In this section we detail some interesting properties of various
bits of equipment which may be found in \cf . \\

\indent{\bf Books:}\inputimage{books} \\
\index{books}
% \index{spells, how to learn}
This is how players can obtain magical spells, sometimes a player can
learn the spell, other times they cannot. The chance depends on the type
of spell, either INT (\incantation s) or WIS (prayers) is used to help
determine the percentage chance that the spell might be learned (see section
\ref{sec:spell_learn} for details).

There are many, many different types of books out there,
as well as being spell books (\wizbook s and prayerbooks), the
following information can
appear in books generated in shops and/or monster treasure hoards$-$
\begin{quote}
        $\bullet$ Compendiums on monsters. Their powers/abilities are
            described as in the \spoiler .

        $\bullet$ Compendiums of \incantation s/prayers by spell Path. Higher
            level texts are more complete in their description of
            available spells.

        $\bullet$ ``Bibles'': various aspects, properties, and characteristics
            of a God/cult are described. Higher level texts
            have more/better information.

        $\bullet$ Compendiums explaining the powers of magic items. Higher
            level texts have more items detailed.

        $\bullet$ Alchemical Formulae.

        $\bullet$ Other randomly generated information.
\end{quote}
Book level is assigned when the book is generated as treasure.
Level is based on the difficulty of the map the book is
generated on. All information is {\em server} specific. \\

{\bf Scrolls and Potions:}\inputimage{potions}\inputimage{scrolls}
\index{scrolls}\index{potions}\\

	Most of these items provide a one-shot use of a spell without
	making the user expend either mana or grace. Scroll use
	depends on the user's {\tt literacy} skill and may fail. Potions
	always work, but are more expensive to buy. Several kinds
	of items are classed as "potions": balms, figurines, and
	dusts. Some potions don't cast spells, but instead raise
	the drinker's stats. Beware cursed potions. They can {\em lower}
	your stats instead! \\

{\bf Wands(Staves)/Rods/Horns:}\inputimage{wands}\inputimage{rods}
\inputimage{horns}\index{wands}\index{rods}\index{horns} \\
	These items provide use of spells. Wands have a limited
	number of charges, while horns and rods will recharge
	(but don't fire as much damage in a small amount of time).
	Horns are used at the overall level of ability of the
	user, while rods and wands cast spells at the item level.\\

{\bf Rings:}\inputimage{rings} \\ \index{rings}
     Many different types, rings can be worn to add/remove different
     immunities, gain/lose spell Paths and alter all types of stats. \\

{\bf Food/Flesh:}\inputimage{food}\inputimage{flesh}\index{food}\index{flesh} \\
	These items provide sustenance. Food is generally more healthy
	to eat, while some flesh items can be sold for good cash. Both
	types may temporarily alter your stats, and/or be poisonous.
	Many flesh items inherit the properties of the monster they
	came from. For example, a ``poisonous'' monster will leave
	behind poisonous flesh. Don't eat it if you know what's good
	for you!!\\

{\bf Weapons/Armour: }\index{armour}\index{weapons}\\
     Tons of items, it is up to you as the player to figure out which work
     better then others. Take a look at weapon/armour weight in the \spoiler\
	to get an idea of how enchanted unidentified items are. \\

{\bf Artifacts:}\index{artifacts}\\
     These are the real treasures of the game. There are more than 20
     artifacts out there, but they are very hard to come by. \\

\section{Encumbrance}\label{sec:encumberance}\index{equipment, encumbrance}\index{encumbrance}
Armour, weapons, shields will encumber a wizard and cause spell
failure.  Light equipment causes no failure at all whereas heavy equipment
causes mondo failures.

The reasoning is that the bulkiness of objects, not their weight exactly, is
what causes failures. So the basic idea of encumbrance is that items get in the
way more than they weigh down. Unfortunately, our only measure of 'getting
in the way' was the weight.

\subsection{How encumbrance is calculated}
Encumbrance points are tallied only from {\em applied} objects. Weapons
give
3x their weight in kg in encumbrance points. Shields give 1/2 their weight
in kg in encumbrance points. Armour gives its weight in encumbrance points.

There's an allowance of encumbrance points which all players get before they
start losing \incantation s, this was about 35-45, not too much.

The formula works like this: You make a roll of 1-200. You compare it to a
failure threshold. This threshold is: encumbrance + \incantation\ {\tt level}
 - caster {\tt level} - 35

For example, lets say a 4th {\tt level} wizard is casting a 5th {\tt level}
\incantation . The wizard is wearing plate mail (100 kg), a 20 kg shield and
wielding a 15 kg weapon. His encumbrance is 100 + 10 + 45 = 155. Thus, his
threshold for failure is 155 + 5 - 4 = 156 or just about 3/4 failure rate.

There is no special bonuses for using magical equipment, although, it is
clear that magical armour and weapons make things better through their weight.

\section{Enchantments}\label{sec:enchant}
\index{equipment, magic}

Some items will have numerical values such as +1, +2, +3, etc.
trailing their names. These {\em magic} values indicate that the item
is enchanted,
and in some way may be better or (if the value is negative) worse
than ordinary runaday items of its kind.

\subsection{Enchanting armour}\index{enchantment, armour}\index{armour}

Enchantment of armour is achieved with the
{\em enchant armour}\inputimage{scrolls}\ scrolls.
Each time you successfully use a scroll, you will
add a plus value, more armour to the piece of equipment and
some fractional amount of weight.

You may only add up to 1 + (overall {\tt level}/10) (rounded down like an
integer) in pluses to any one piece of armour. How much
armour value you add to the item is also dependent on your
overall {\tt level}. You may never enchant a piece of armour to
have an armour rating greater than their overall {\tt level} or 99.

\subsection{Enchanting weapons}\index{enchantment, weapons}
\index{weapons}

This is done via a series of scrolls\inputimage{scrolls} that you
may find or buy in
shops. The procedure is done in two steps. Use the {\em prepare weapon} scroll
to lay a magic matrix on your weapon. Then use any of the other
scrolls to add enchantments you want. Note that some of these scrolls will
also require a ``sacrifice'' to be made when they are read. To sacrifice
an object just stand over it when you read the weapon scroll. Scrolls
for weapon enchantment are: \\

{\bf Prepare weapon} \\
Diamonds are required for the sacrifice. The item
can be enchanted the square root of the number of diamonds sacrificed. Thus,
if 100 diamonds are sacrificed, the weapon can have 10 other enchant scrolls
read. \\

{\bf Improve damage} \\
     There is no sacrifice. Each scroll read will increase the damage by 5
     points, and the weight by 5 kilograms. \\

{\bf Lower (Improve) Weight} \\
     There is no sacrifice. Each scroll read will reduce the weight by 20\%.
     The minimum weight a weapon can have is 1 gram. \\

{\bf Enchant weapon} \\
     This does not require any sacrifices, and increases the magic by 1. \\

{\bf Improve Stat} (ie, Strength, Dexterity, etc) \\
     The sacrifice is the potion\inputimage{potion} of the same type
as the ability to be
     increased (ie, Improve Strength requires strength potions). The number
     of potions needed is the sum of all the abilities the weapon presently
     gives multiplied by 2. The ability will then be increased by 1 point.
     Thus, if a sword is Int +2 and Str +2, then 8 potions would be needed to
     raise any stat by one point. But if the sword was Int +2, Str +2, and Wis -2,
     then only 4 potions. A minimum of 2 potions will be needed. \\

{\bf WARNING:} something to keep in mind before you start enchanting
like crazy$-$you can only use a weapon that has 5 + 1 enchantments
for every 5 levels of
{\tt physique} experience you possess. So, a character with
10th level in the {\tt physique} experience category may only be able to
use a weapon with a maximum of 7 enchantments!

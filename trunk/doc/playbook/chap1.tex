
\chapter{Introduction}

\section{What is \cf\ ? }
 
In the words of its original author Frank Tore Johansen :
\begin{quote}
``\cf\ is a multi-player graphical arcade and adventure game made for the
X Windows systems environment. It has certain flavours from other games,
especially \gauntlet\ and the rogue-like  games (mainly \hack , \moria ,
\angband , and \ragnarok ). Any number of players can move around in their
own window, finding and using items and battling monsters. They can
choose to cooperate or compete in the same {\em world}.'' 
\end{quote}
In the years of development that have followed, \cf\ has grown to 
encompass over 150 monsters, $\sim$3000 maps to explore, an elaborate
magic system, over 15 character types, a system of skills, and
many, many artifacts and treasures. 

\cf\ is freely distributed under the GNU license and the code may be
obtained from a number of FTP sites (see section \ref{sec:obtain}).
{\bf Take note:} you will need at least an ANSI C compiler (i.e. {\tt gcc}) 
to build the game. A screen shot of the \cf\ display appears in Figure \ref{fig:dis}.

\begin{figure}   
\mongovaryboth fig/screen_dia.ps 3.5in 5.5in
\caption{Screen shot of the game display which comprises of six
different windows. \label{fig:dis}}
\end{figure}

\cf\ is currently being maintained by Mark Wedel ({\tt mwedel@sonic.net}).
Crossfire is uses the client server model - clients for unix (using
gtk or straight X11) and windows are readily available.  SDL client
is being worked on.

To be notified of new releases of \cf\ ,
subscribe to the announcement mailing list
(see section \ref{sec:mlist}) to get notified of updates.

\subsection{What is included in this document?}  

This document is a guide intended to focus on the game-play aspects of \cf . 
The original intention of this text was to 
help aspiring players create and play a character and more quickly initiate
them into the intricacies of \cf . But, as things went along, many of the 
older docs began to be incorporated and lots of good reference material 
slipped in; this document may also be a good resource for more experienced 
players.

Having said all of that, lets point out that each release contains
a fair amount of documentation and the \playbook\ is definitely not the 
last word! Certainly, every player will want to obtain the \spoiler , 
and while this may sound flippant, its still true\emdash a good place to 
start tracking down much of the information you may want is the 
README document in the top directory of your release.


\subsection{Getting started: beginning players}\index{beginning players} 

First-time players may want to skim over sections 
\ref{sec:char_attr}, \ref{sec:char_gen}, \ref{sec:basic},
and \ref{sec:first} before playing the game. Don't be daunted by 
the apparent complexity of \cf \emdash in reality \cf\ is quite easy 
to play and character generation
is simple. Later on, when you have some experience
playing the game, you may wish to go back and re-read skimmed material
and expand your knowledge of the unread sections. 

If you just want to damn the torpedoes and ``roll''\footnote{This archaic 
term comes from paper and pencil role-playing games which used dice in 
the character generation process.} up a character quickly then proceed 
to section \ref{sec:char_gen} in this text. You may {\em still} 
find section \ref{sec:first} helpful reading. 


\section{\cf\ Mailing lists}\index{mailing lists}\label{sec:mlist} 

Two mailing-lists exist; the first is for discussing bugs and
development while the second is for just announcing new versions.
If you want to join or leave any of the lists, send mail to
{\tt crossfire-}{\tt request@}{\tt ifi.uio.no} with the subject ``{\tt subscribe}'',
``{\tt unsubscribe}'',``{\tt subscribe announce}'' or ``{\tt unsubscribe 
announce}''.  If you use the subject ``{\tt subscribe}'' or 
``{\tt unsubscribe}'' you will (un)subscribe to {\em both} lists.

To send messages directly to the list, mail {\tt crossfire@ifi.uio.no}.
While the development/bug list is not supposed to be used for helping
players solve problems, sometimes a knowledgeable reader of the 
list can be tempted into answering your question.

An archive of old messages can be found at {\tt ftp.ifi.uio.no} in the
directory {\tt /pub/crossfire/archive}. 

\section{Obtaining \cf } \label{sec:obtain}
 
FTP sites\index{FTP sites} where you can look for the latest versions are: \\
 
\noindent{\tt ra.pyramid.com:/pub/crossfire} (129.214.1.102) \\
{\tt ftp.ifi.uio.no:/pub/crossfire} (129.240.64.44) \\
{\tt ftp.real-time.com/pub/games/crossfire} (206.10.252.12) \\
{\tt ftp.cs.city.ac.uk:/pub/games/crossfire} (138.40.91.9)\\
{\tt ftp.sunet.se:/pub/unix/games/crossfire} (130.238.127.3) \\
{\tt ftp.cs.titech.ac.jp:/pub/games/crossfire} (131.112.90.201) \\
 
Please use the one nearest to you.  ``.se'', ``.no'' or ``.uk'' in Europe,
``.au'' in Australia, ``.net'' in North America, and ``.jp'' in Asia.

\section{DOCUMENT CREDITS}

B.T. provided the bulk of the written material; K.E. and 
(to a lesser extent) J.K provided expert editing and typesetting advice. 
The \playbook\ uses material borrowed from older documentation written 
by the following people (in rough order of borrowing magnitude): 
\begin{quote}
Sam Mackrill (lots of bits from the old FAQ) \\
Laurent Wacrenier \\
Kjetil T. Homme \\
Lars H. B. Olafsen \\
Peter Mardal \\
Rupert G. Goldie \\
Frank T. Johansen
\end{quote}
If I left anyone out, please send me some email! \\ 

\noindent{Brian Thomas} \\ {\tt thomas@astro.psu.edu} \\


\chapter{Playing \cf}

\section{Basics}\label{sec:basic}

In this section, several basic bits of information are detailed in 
a concise way in rough order of importance.
Various pointers to other sections of this document will help you to 
round out your knowledge if you want to. All of the available player
commands are concisely explained in appendix \ref{app:commands}. You
can always get a summary of available commands while playing the game;
hit ``{\tt ?}'' for help. \\

\subsection{How to do simple stuff} \index{How to do simple stuff}

\subsubsection{Move around and attack}\index{commands, movement}\index{movement}\index{combat}\index{running} 
Movement is accomplished with the mouse, or
with the same keys that some rouge-like computer games use. To move using
the mouse, position the cursor over a square you wish to move to
in the view window 
then click the right hand button. If you want to use the keys, here's a 
simple diagram of where the various movement keys will take you: \\ 
\begin{center}
\begin{tabular}{ccccc} 
{\tt y} &  & {\tt k} &  & {\tt u} \\
  & $\nwarrow$  & $\uparrow$  & $\nearrow$ &   \\
{\tt h} & $\longleftarrow$ & .  & $\longrightarrow$ & {\tt l} \\ 
  & $\swarrow$  & $\downarrow$  & $\searrow$ &   \\
{\tt b} &  & {\tt j} &  & {\tt n} \\
\end{tabular}
\end{center}
The ``{\tt .}'' refers to yourself; you don't move anywhere when you
press it. 
In order to ``run'' in a particular direction (i.e. move continuously 
without having to repeatedly punch the key) hold down the control
key then hit any movement key or the right hand mouse button to
start moving. Release the {\tt $<$control$>$} key when you wish to stop running. 

If you move into something, you will attack it. This means walls,
doors, and monsters will be damaged if you hit them. Players and
friendly monsters may also be attacked in this way, but only if 
you set the peaceful flag to ``off''. To learn more about the combat
system see section \ref{sec:combat}. \\ 

\subsubsection{Pickup stuff}\index{commands, pickup}\index{picking up items}
To pickup items, move over the item, then either hit the ``{\tt ,}'' key 
or move the cursor over to the look window, position it over the desired
item and click the left mouse button. You will see the item appear in your 
inventory window. If you pick up too much stuff, you won't be
able to see it all at once. Use the ``{\tt $<$}'' and ``{\tt $>$}'' keys 
to rotate through the inventory list. \\ 

\subsubsection{Applying stuff: wear armour, wield a weapon, eat, and so on.}\index{commands, apply} 
Most of the time, in order to manipulate or ``{\tt apply}'' items you have 
to be holding them. Move the cursor over to the desired item in the 
inventory window. By using the middle button on the mouse, you may 
toggle the status (ie between ``applied'' or ``unapplied'') of items. 
Note that 
some items when applied will be used up (they disappear from the 
inventory window). Examples of these kind of
items include ``food''\inputimage{foodone}, ``potions''\inputimage{potion}, 
and ``scrolls''\inputimage{scrolls}. 
To learn more
about the uses of various items see chapter \ref{chap:equip}. \\ 

\subsubsection{Shoot a ranged weapon}\index{commands, fire}\index{bows}
\index{wands}\index{rods}\index{horns} 
Ranged weapons include bows\inputimage{bows}, wands\inputimage{wands}, 
rods\inputimage{rods}, or horns\inputimage{horns}. Apply the desired
weapon, then check to see that the {\tt Range:} slot in the status window
indicates that item is ``readied'' (yes...you can have something applied but
not readied). If its not ready, use either the plus or minus keys to 
rotate though all of the slots. Once readied, use the ``{\tt $<$shift$>$}'' key
followed by a movement key to fire the object in that direction. Alternatively,
place the cursor in the view window, then hit the middle mouse button to fire. \\ 

\subsubsection{Enter a building or boat.}\index{commands, apply}
\index{entering buildings}\index{movement}\inputimage{guild} 
Move over on top of the desired structure. Then hit either middle mouse
button while the cursor is on the icon of the structure in the look window, 
or hit the {\tt A} key. If there is a link to a map drawn of the ``inside'' 
you will be taken there. If no link exists, you will get the message
``{\tt the $<$structure$>$ is closed.}''. \\ 

\subsubsection{Use a skill}\index{commands, ready\_skill}
\index{skills, how to use} 
In order to use a skill, it must first be readied. You can ready any skill 
you have with the {\tt ready\_skill} command. Also, some skills will 
auto-matically be readied when you undertake certain
actions. For example, if you run into a hostile monster with a wielded weapon
the {\tt melee weapons} skill is readied. A ready skill will show up in the 
stat window in the {\tt Range:} slot. If a skill doesn't appear in the slot, rotate 
the range slot to check for the skill. When a skill is readied, the range slot will
appear as ``{\tt Skill: $<$skill$>$}'' (otherwise it appears as 
``{\tt Skill: none}''). 
To use the skill, make a ``ranged attack'' (ie hit the same keys or
mouse button as for firing a wand). To learn more about the skills
system see chapter \ref{chap:skills}. \\ 

\subsubsection{Cast a spell}\index{spells, how to use}\index{commands, cast} 
\index{talisman}\index{holy symbol}
In order to ``{\tt cast}'' spells (either \incantation s or prayers), you must have 
the skills of {\tt \spellcasting\ } (\incantation s) or {\tt praying} (prayers). 
Possession 
of a ``talisman''\inputimage{talisman}\ or a ``holy symbol''\inputimage{holysymbol}\ 
will also allow you to respectively {\tt cast} \incantation s or prayers). You can 
only {\tt cast} those spells you have {\em learned}. Issue the meta-command 
{\tt `cast $<$spell$>$} to ready a spell in the {\tt Range:} slot. To 
``fire'' the spell, make 
a ranged attack. Note! if you don't have enough mana a\ina\ \incantation\ 
{\em} will fail.
If you don't have enough grace a prayer {\em may} work. For more information
concerning the magic system see chapter \ref{chap:magic}.

\subsection{Saving the game and ending the \cf\ session:}\index{commands, quit}
\index{commands, save}\index{quitting}\index{saving} 
 
The {\tt save} command is to provide an emergency backup in case of a game crash.
To save your player at the end of your game session you must find a ``Bed to 
reality''\inputimage{savebed}, move your player over it and 
{\tt apply} it (``{\tt A}''). These beds can usually be
found in the inns and
taverns dotted around the maps (especially in cities). This prevents you
just saving anywhere and forces you to finish what you are doing and return
somewhere safe.

\subsection{About NPCs}\index{commands, say}\index{commands, \"}
\index{NPC}\index{talking} 
{\em N}on {\em P}layer {\em C}haracters are special
``monsters'' which have ``intelligence''. Players
may be able to interact with these monsters to help solve puzzles and find
items of interest. To speak with a monster you suspect to be a NPC, simply
move to an adjacent square to them and push the double-quote, ie. {\tt "}. Enter
your message, and press {\tt $<$return$>$}. You can also use the meta-command 
{\tt 'say} if you feel like typing a little extra.

Other NPCs may not speak to you, but display intelligence with their
movement. Some monsters can be friendly, and may attack the nearest of your
enemies. Others can be ``pets'', in that they follow you around and help you
in your quest to kill enemies and find treasure.


\subsection{Important special items}

There are many special items which can be found in \cf , of those shown
below, the most important may be the signs. \\ 

$\bullet$ {\bf Signs:}\index{signs}\inputimage{sign} 

Signs often have messages that might clue you in on quests and puzzles
or even refer you to NPCs. It is your job as a player to make sure you
read every sign to make sure you do not miss things. \\ 

$\bullet$ {\bf Handles and Buttons:}\index{handles}\index{buttons}
\inputimage{handbutt} 
 
These items can often be manipulated to open up new areas of the map.
In the case of handles, all a player must do is apply the handle. In
the case of buttons, the player must move items over the button to
"hold" it down. Some of the larger buttons may need very large items to
be moved onto them, before they can be activated. \\ 
 
$\bullet$ {\bf Gates and locked doors:} \inputimage{gatedoor}
\index{doors}\index{gates}
 
Gates are often tied to a handle or button and can only be opened by
manipulating the the handle/button. Doors that are locked can either be
smashed down by attacking the door, by using keys\inputimage{keys}\ 
which can be found
throughout the game, or by picking the lock. \\ 
 
$\bullet$ {\bf Pits:}\inputimage{pit} 
\index{pits}
 
Pits can be doorways to new areas of the map too, but be careful, for
you could fall down into a pit full of ghosts or dragons and not be
able to get back out! \\ 
 
$\bullet$ {\bf Break away walls:}\inputimage{br_wall} 
\index{walls}
 
Are a common occurrence in \cf . These type of walls can be
"destroyed" by attacking them. Thus, sometimes it may be worth a
player's time to test the walls of a map for ``secret doors''. \\ 
 
$\bullet$ {\bf Fire walls:} \inputimage{fr_wall} 
 
Will shoot missiles (including bullets, lightning, etc.) at players. 
Some firewalls can be destroyed while others cannot. \\ 
 
$\bullet$ {\bf Spinners and Directors:} \inputimage{dir_spin}
\index{spinners}\index{directors}
 
These odd items will change the direction of any item flying over them,
such as missile weapons and spells. \\ 


\section{Matters of life and death} 

\subsection{Attack system}
\label{sec:combat}
\index{combat}\index{attacktypes}\index{slaying}\index{Dam}\index{Wc}\index{Ac} 

Every time you make an attack, your attack is classified with one or more 
``attacktypes''. For example, an attack made with a ordinary sword
results in the attack being made with the attacktype of ``physical''. 
For another example, if a Mage attacks with a fireball spell the 
attack is made with the ``magic'' and ``fire'' attacktypes. In
similar fashion, a defender may be protected, vulnerable, or immune  
to any attacktype. \\

\noindent{Here} is a summary of the attack system; in fact, its a 
bit more complicated. \\ 

\subsubsection{Hitting an opponent}
\indent{Several} quantities are involved in determining whether an attack will
hit its target.  The attacker will hit if his {\tt Wc} is less than or equal
to the 
defender's {\tt Ac} + 1D20\footnote{a random number between 1 and 20} or 
if the D20 gives a 20 (remember, both {\tt Wc} and {\tt Ac} improve as
their value drops {\em lower}).\\ 

\subsubsection{Damaging a hit opponent}
\indent{Damages} are randomly generated, with the magnitude of the random number
being based on the attacker's {\tt Dam} stat. 
If defender is immune to an attacktype
in the attack, he receives no damage, if he's protected he receives
half damage, if he's vulnerable to this kind of
attacktype, he receives double damage. For physical attacks, the percent 
of your {\tt Arm} value is subtracted from incoming damage.
Some magic weapons can ``slay'' various races of creatures. If the weapon 
slays the defender, the attack damage is tripled. 


\subsection{Experience}\label{sec:experience}\index{experience}
\index{experience, categories}\index{level, overall}\index{level}

Accumulation of experience will result in increasing the {\tt level} of the 
player's character. In a \cf\ game where skills are {\em not} present, 
experience is only gained for removing traps and killing monsters.
A player will gain a new {\tt level} when their 
experience total reaches a new amount in the hierarchy shown
in table \ref{tab:exp_level}. 

In the skills-based game, {\em several} kinds of experience exist. The 
{\tt score} represents the ``{\em overall}'' proficiency of the player 
and is the 
{\em sum} of all the differing kinds of experience possessed.

The player accumulates experience into various ``{\em experience 
categories}'' according to their actions.
Another way of putting this is that you become better
at what you do most often and most successfully. For example,
a player who kills monsters with \incantation s gains experience in 
casting \incantation s. A player who steals from creatures often will
get better at stealing, and so on.  

Each experience category will have a {\tt level} assigned to it based on 
the amount of experience accumulated in it (using the same schedule
 shown in table 
\ref{tab:exp_level}). Use the {\tt skills} command to 
investigate which skills your character has and to see the {\tt level} of 
ability you have in each experience category. See chapter 
\ref{chap:skills} for more information about skills and skill-based
experience.

As an aside, monsters {\em also} are assigned a {\tt level} of proficiency
and may gain experience. The main way which monsters gain experience
is by {\em killing} players! Beware going after a monster that has
killed several players, it will be much more difficult to challenge!

\begin{table}
\begin{center}
\caption{Relationship between experience and {\tt level} for the first 
10 levels. \label{tab:exp_level}}
\vskip 12pt
\small
\begin{tabular}{|rl|}
\hline
{\tt Level}	& \multicolumn{1}{c|}{Experience} \\ \hline\hline 
\input{levels}
... & ... \\
\hline
\end{tabular}
\end{center}
\end{table}
 
\subsection{Calculation of selected secondary stats}\label{sec:stat_calc} 
\index{stats, secondary}

Both the primary stats and {\tt level} of the character will influence 
the secondary stats given below. In the skills-based game, the 
appropriate experience category\footnote{categories
are given for the default settings, this can be changed by the 
server administrator} is identified for purposes of determining 
which {\tt level} is used in the calculation. In the non-skills
game, the {\em overall} {\tt level}\index{level, overall} is always used.

Refer to table \ref{tab:pri_eff} if a quantity in a calculation is
left unexplained. 

\subsubsection{Weapon class ({\tt Wc})}\index{Wc}\index{stats, Wc} 
The {\tt Wc} calculation is:
\begin{quote}
class {\tt Wc} - {\tt thaco}(STR)\index{thaco} - weapon {\tt Wc} - {\tt level} - 1 every 6 {\tt level}s  
\end{quote}
where {\tt thaco} is found using the STR stat on table \ref{tab:pri_eff}, 
the weapon {\tt Wc} 
can be determined from the \spoiler\ and the {\tt level} is taken from the 
{\tt physique} experience category. 

\subsubsection{Weapon Speed}\index{stats, weapon speed} 
\index{weapon speed}
The calculation for weapon speed is quite complex (but here it is!).
The value for the weapon speed is:
\begin{quote}
{\tt speed}/({\tt LF} * {\sl NastyFactor} * {\sl LessNastyFactor})
\end{quote}
where {\tt LF} (``level factor'') is a number between 0.8 and 1.2 that 
increases with the {\tt level} as (4+{\tt level})/(6+{\tt level})*1.2,
(the {\tt level} is taken from the overall {\tt level}),
and both {\sl NastyFactor} and {\sl LessNastyFactor} are calculated below.

The value of {\sl NastyFactor} is:
{\small
\begin{quote} 
2/3 + {\tt MaxC}/363 - ((0.00167*{\tt WpnW})/{\tt MaxC}) + ({\tt speed}/5) + ((Dex-14)/28) 
\end{quote}
}
\noindent{where} {\tt MaxC} is the maximum carrying 
limit of the character, {\tt WpnW} is 
the weapon weight. Weapon weight can be determined from examining the given 
weapon (ie look at the number to the right side of the weapon icon in the 
inventory window).

The value of {\sl LessNastyFactor} is:
\begin{quote} 
2 - ((WpnF - {\tt magic}/2)/10) 
\end{quote}
where {\tt WpnF} is the ``weapon factor'' and {\tt magic} is the 
value of the enchantment on the weapon (i.e. $+$1, $+$2, etc.).
Weapon factor\index{weapon factor} is given for some weapons
in table \ref{tab:weap_factor}. \\

\begin{table}
\begin{center}
\caption{Weapon factor ({\tt WpnF}) for selected weapons}\label{tab:weap_factor}  
\small
\vskip 12pt
\begin{tabular}{|crlc|} \hline 
	& & & \\
       &  1.6 & nunchacu &  \\
       &  1.5 & dagger & \\  
       &  1.4 & Sting, katana, shortsword, taifu & \\
       &  1.3 & Belzebub's sword, Darkblade, Excalibur & \\
       &  1.2 & Firebrand, Frostbrand, Mjoellnir, Mournblade, &  \\
       &      & Stormbringer, broadsword, falchion, light sword & \\
       &      & long sword, quarterstaff, sabre, sword, trident & \\
       &  1.1 & Deathbringer, Demonbane, Dragonslayer, Holy Avenger & \\
       &      & club, hammer, mace, unicorn horn & \\
       &  1.0 & axe, stonehammer & \\
       &  0.9 & Skullcleaver, morningstar, stake, stoneaxe & \\
       &  0.8 & large morningstar & \\
       &  0.5 & Bonecrusher, Gram, shovel & \\
       &  0.3 & large club &  \\    
       &  0.1-0.4 & chair & \\ 
       &  0.1 & bed & \\ 
	& & & \\ \hline 
\end{tabular}
\end{center}
\end{table}


\subsubsection{Damage ({\tt Dam})}\index{stats, Dam}\index{damage}

Current weapon, character class 
(table \ref{tab:char_cls}) and STR ({\tt DmB} in table \ref{tab:pri_eff}) 
all effect the value of {\tt Dam}. The calculation for {\tt Dam} is:
\begin{quote}
Class bonus $+$ {\tt DmB} $+$ Weapon damage $+$ Skill damage 
\end{quote}
In addition, for every 4 levels of fighting expertise ({\tt physique} 
experience category level) 1$+$({\tt Dmb}/5)) is added to {\tt Dam}. 
Some skills (namely the hand-to-hand, or martial arts skills) 
can add significant damage to the overall total. The amount
that is added depends on the user's {\tt level} in that skill.
Note that weapon damage is automatically added to your damage
rating in the stat window when you wield any weapon.


\subsubsection{Armour class ({\tt Ac})}\index{stats, Ac}\index{Ac}

For characters that cannot wear armour (ex. Fireborn), their 
{\tt Ac} can improve as their score increases. The calculation for {\tt Ac} is then: 
\begin{quote}
Class {\tt Ac} $-$ {\tt level}/3
\end{quote}
The value of the character {\tt Ac} may decrease to the value of -10.
After that, no more improvement is made regardless of earned
experience. The overall {\tt level}\index{level, overall} is 
{\em always} used for this calculation.


\subsubsection{Hit points ({\tt Hp})}\index{hit points}\index{stats, Hp} 

A character will gain the following number of {\tt Hp} for {\em each} new
overall {\tt level} gained between levels 1$-$10:
\begin{quote}
({\tt HpB})/2 + D4 + D4 + 1 + 1/every even {\tt level} {\em if} {\tt HpB} is even.
\end{quote}
Where D4 is a random number between 1 and 4 and you always get 1 hit point
gained per {\tt level}, regardless of {\tt HpB}. 
For levels after 10th, {\tt Hp} grow 2 per {\tt level}.
The overall {\tt level} provides the correct value for {\tt level} in 
the calculation.
 

\subsubsection{Mana ({\tt Sp})}\index{mana}\index{stats, mana} 

A character will gain the following number of {\tt Sp} for {\em each} new
{\tt magic} experience category {\tt level} gained between levels 1$-$10:
\begin{quote}
({\tt MgB}(POW))/2 + D3 + D3 + 1 every even {\tt level} {\em if} {\tt MgB} is even.
\end{quote}
Where D3 is a random number between 1 and 3 and you always get 1 mana gained
per new {\tt level}. For {\tt magic} levels past the 10th, mana grows 2 per level.
 

\subsubsection{Grace ({\tt Gr})}\index{stats, grace}\index{grace} 

A character will gain the following number of {\tt Gr} for {\em each} new
{\tt wisdom} experience category {\tt level} gained between levels 1$-$10:
\begin{quote}
({\tt MgB}(WIS) + {\tt MgB}(POW))/8 + D3
\end{quote}
Where D3 is a random number between 1 and 3 and you always get 1 grace gained
per new {\tt level}. For levels past the 10th, 
grace grows 1 per {\tt wisdom} {\tt level}.

\subsection{Death $---$ the ultimate penalty?}\label{sec:death}

Should your hit point total drop below 0 you will die\inputimage{gravestone}. 
What happens next depends on how the game is compiled. If the 
{\tt NOT\_PERMDEATH} option is being 
used (this is the default) then your character (and his all of his 
{\em carried}
equipment at the instant of death) will appear in the base map (the one
you first started out in). 

For each time you die, your character loses 20\%
of their experience (in all categories) and a random primary stat decreases by 1. 
At low levels, the stat loss is bad news, while it is the experience loss that
{\em really} hurts at those high levels. You may still re-gain lost stats (up to your
natural limit) by drinking potions.

If the server administrator has made death permanent, there is still hope
for your character. When the {\tt RESURRECTION} option is enabled (the default option
when {\tt NOT\_PERMDEATH} is {\em not} being used) other characters may bring you back
from the beyond. There are 3 spells that are capable of doing this, but
remember to retrieve the corpse\inputimage{corpse}\ of the character 
that is to be resurrected! 

 
\section{Some advanced stuff}

\subsection{Some useful advanced commands}

\subsubsection{Meta-command ({\tt '}) }\index{meta-command}\index{commands, meta} 
Any command listed by the use of help ({\tt ?} key) can be
used by spelling it out after the meta-command is issued. For example, you
can use the command {\tt say} either by hitting the double-quote ({\tt "} key) or
by issuing the command string {\tt 'say}. 

\subsubsection{Binding commands ({\tt 'bind} and {\tt 'unbind})}\index{binding}\index{unbinding} 
\index{commands, bind}
\index{commands, unbind}
You may bind any key with a complex command. For example, you could
bind the use of the meditation skill to the key ``{\tt m}''. To do this first
type:
\begin{quote}
{\tt 'bind use\_skill meditation } 
\end{quote}
then press {\tt $<$return$>$}. The game will then ask you for a key to bind the
command to, you then hit {\tt m}. You can also re-bind this key to something
else later if you wish too. Issue the command:
command:
\begin{quote}
{\tt 'unbind reset } 
\end{quote} 
to totally reset bindings on keys. 

\subsubsection{Pick up toggle ({\tt @})}\index{picking up items}\index{commands, pickup}
This allows you to change your pickup status. Eight different modes
for pick up exist: ``don't pick up'',``pick up 1 item'', ``pick up 1 item and
stop'', ``stop before picking up'', ``pick up all items'', pick up all items
and stop'', ``pick up all magic items'', ``pick up all coins and gems''.
Whenever you move over a pile of stuff your pickup mode controls 
if and what you collect. You can always pickup stuff using the pickup 
command ({\tt ,}) regardless of your current pickup mode. 
 
\subsubsection{Invoke ({\tt 'invoke})} \index{commands, invoke} 
A useful way to quickly cast both \incantation s and prayers is via the invoke
command. To use it effectively, {\tt bind invoke $<$spell$>$} to any 
key. Then, when that binded key is pressed, your character will cast that magic
in the direction they're currently facing.

\subsection{Playing with other people}

As a general rule, other \cf\ players will prefer to co-operate or
at least leave each other alone. If you go about killing other player's
characters you may not only risk their continued wrath, but the anger of
the server administrator as well. Check out the house rules before you 
start slaying players.

\subsubsection{Useful multi-player commands}

Here are some useful commands for playing with other players: \\

\noindent{\bf\tt shout}\index{commands, shout} \\
This meta-command will broadcast your message to every player currently
logged in. The say command only sends messages to players who share your
current map. Invoke this command just like {\tt say}.\\ 

\noindent{\bf\tt  who}\index{commands, who} \\
This will give you a listing of all the current players and the maps
they are on currently. Invoke this command as {\tt 'who}.\\ 

\noindent{\bf\tt tell}\index{commands, tell} \\
Will send your message only to the player indicated. It is invoked as:
{\tt 'tell $<$character name$>$ $<$message string$>$}.\\ 

\subsubsection{The simple party system}\index{party system}\index{commands, party} 
If the {\tt SIMPLE\_PARTY\_SYSTEM} has been enabled on your server you can 
use this to adventure with other players. All experience gained by 
members of the party is split equally, and in addition, party members 
are always peaceful towards each other. Here's 2 useful party commands:\\

\noindent{1)} To form a party issue the command: 
\begin{quote}
{\tt 'party form $<$party name$>$ }  
\end{quote}
2) To join a party, type:
\begin{quote}
{\tt 'party join $<$party name$>$ } 
\end{quote}
To see all of the options, type {\tt 'party help}.



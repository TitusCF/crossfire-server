% 
% 
  
%\documentstyle[local,A4,longtable,psfig,12pt]{report}  

\documentclass[12pt, a4paper]{report}
\usepackage{local}
\usepackage{longtable}
\usepackage{epsfig}
\usepackage[latin1]{inputenc}
\usepackage[T1]{fontenc}
\usepackage[english]{babel}

\def\inputimage#1{ \input{#1}}

% uncomment in the following line if you
% DONT want to have icons in your text. 

% \def\inputimage#1{}


% TEXT DEFINITIONS
\def\etal{{\it et al.}}
\def\cf{{\sl Crossfire}}
\def\playbook{{\sl Crossfire} {\bf Player's Handbook}}
\def\spoiler{{\sl Crossfire} {\bf Spoiler}}
\def\spellcasting{wizardry} % name of the skill that allows casting incantations
\def\wizbook{grimore} % name of the text you learn incantations from 
\def\divinemagic{divine magic}
\def\wizardry{wizardry}
\def\alchemy{{\tt alchemy}} % name of the spell to use alchemy code
\def\mage{magery} % mage experience category name 
\def\priest{wisdom} % priest experience category name 
\def\incantation{incantation}
% this is the case a/an for 'incantation' def
\def\ina{n}
% using this caus Klaus yelled at me ;)
\def\emdash{{\em $-$}}
% here are some names
\def\angband{{\sl Angband}}
\def\ragnarok{{\sl Ragnarok}}
\def\gauntlet{{\sl Gauntlet (TM)}}
\def\rogue{{\sl Rogue}}
\def\hack{{\sl NetHack}}
\def\moria{{\sl Moria}}
\def\sngc#1{\multicolumn{1}{c}{#1}}
\def\snglc#1{\multicolumn{1}{|c}{#1}}
\def\sngLc#1{\multicolumn{1}{|c|}{#1}}
\def\tplc#1{\multicolumn{3}{c|}{#1}}
  

%\pssilent
			 
\pagestyle{empty}

\title{Player's Handbook for Crossfire} 
\author{Brian Thomas\footnote{thomas@astro.psu.edu}}
  
\makeindex

\begin{document}	  
\pagestyle{plain}
\renewcommand{\thepage}{\roman{page}}

% TITLE PAGE
\thispagestyle{empty}
\dblesp

%
\noindent
\begin{center}
%
{\large \bf Player's Handbook for Crossfire}\\[0.5 cm]
\input{version} \\[1.0 cm]
%
compiled, edited, and written by\\[0.5 cm]
Brian Thomas, Klaus Elsbernd, and John W Klar \\
%
\end{center}


\snglsp
%\dblesp

% CONTENTS
\tableofcontents
% \listoffigures
\listoftables

\pagebreak

\sloppy
\setcounter{page}{1}
\renewcommand{\thepage}{\arabic{page}}

% CHAPTERS

\chapter{Introduction}

\section{What is \cf\ ? }

In the words of its original author Frank Tore Johansen :
\begin{quote}
``\cf\ is a multi-player graphical arcade and adventure game made for the
X Windows systems environment. It has certain flavours from other games,
especially \gauntlet\ and the rogue-like  games (mainly \hack , \moria ,
\angband , and \ragnarok ). Any number of players can move around in their
own window, finding and using items and battling monsters. They can
choose to cooperate or compete in the same {\em world}.''
\end{quote}
In the years of development that have followed, \cf\ has grown to
encompass over 150 monsters, $\sim$3000 maps to explore, an elaborate
magic system, over 15 character types, a system of skills, and
many, many artifacts and treasures.

\cf\ is freely distributed under the GNU license and the code may be
obtained from a number of FTP sites (see section \ref{sec:obtain}).
{\bf Take note:} you will need at least an ANSI C compiler (i.e. {\tt gcc})
to build the game. A screen shot of the \cf\ display appears in Figure \ref{fig:dis}.

\begin{figure}
\mongovaryboth fig/screen_dia.ps 3.5in 5.5in
\caption{Screen shot of the game display which comprises of six
different windows. \label{fig:dis}}
\end{figure}

\cf\ is currently being maintained by Mark Wedel ({\tt mwedel@sonic.net}).
Crossfire is uses the client server model - clients for unix (using
gtk or straight X11) and windows are readily available.  SDL client
is being worked on.

To be notified of new releases of \cf\ ,
subscribe to the announcement mailing list
(see section \ref{sec:mlist}) to get notified of updates.

\subsection{What is included in this document?}

This document is a guide intended to focus on the game-play aspects of \cf .
The original intention of this text was to
help aspiring players create and play a character and more quickly initiate
them into the intricacies of \cf . But, as things went along, many of the
older docs began to be incorporated and lots of good reference material
slipped in; this document may also be a good resource for more experienced
players.

Having said all of that, lets point out that each release contains
a fair amount of documentation and the \playbook\ is definitely not the
last word! Certainly, every player will want to obtain the \spoiler ,
and while this may sound flippant, its still true\emdash a good place to
start tracking down much of the information you may want is the
README document in the top directory of your release.


\subsection{Getting started: beginning players}\index{beginning players}

First-time players may want to skim over sections
\ref{sec:char_attr}, \ref{sec:char_gen}, \ref{sec:basic},
and \ref{sec:first} before playing the game. Don't be daunted by
the apparent complexity of \cf \emdash in reality \cf\ is quite easy
to play and character generation
is simple. Later on, when you have some experience
playing the game, you may wish to go back and re-read skimmed material
and expand your knowledge of the unread sections.

If you just want to damn the torpedoes and ``roll''\footnote{This archaic
term comes from paper and pencil role-playing games which used dice in
the character generation process.} up a character quickly then proceed
to section \ref{sec:char_gen} in this text. You may {\em still}
find section \ref{sec:first} helpful reading.


\section{\cf\ Mailing lists}\index{mailing lists}\label{sec:mlist}

Two mailing-lists exist; the first is for discussing bugs and
development while the second is for just announcing new versions.
If you want to join or leave any of the lists, send mail to
{\tt crossfire-}{\tt request@}{\tt ifi.uio.no} with the subject ``{\tt subscribe}'',
``{\tt unsubscribe}'',``{\tt subscribe announce}'' or ``{\tt unsubscribe
announce}''.  If you use the subject ``{\tt subscribe}'' or
``{\tt unsubscribe}'' you will (un)subscribe to {\em both} lists.

To send messages directly to the list, mail {\tt crossfire@ifi.uio.no}.
While the development/bug list is not supposed to be used for helping
players solve problems, sometimes a knowledgeable reader of the
list can be tempted into answering your question.

An archive of old messages can be found at {\tt ftp.ifi.uio.no} in the
directory {\tt /pub/crossfire/archive}.

\section{Obtaining \cf } \label{sec:obtain}

FTP sites\index{FTP sites} where you can look for the latest versions are: \\

\noindent{\tt ra.pyramid.com:/pub/crossfire} (129.214.1.102) \\
{\tt ftp.ifi.uio.no:/pub/crossfire} (129.240.64.44) \\
{\tt ftp.real-time.com/pub/games/crossfire} (206.10.252.12) \\
{\tt ftp.cs.city.ac.uk:/pub/games/crossfire} (138.40.91.9)\\
{\tt ftp.sunet.se:/pub/unix/games/crossfire} (130.238.127.3) \\
{\tt ftp.cs.titech.ac.jp:/pub/games/crossfire} (131.112.90.201) \\

Please use the one nearest to you.  ``.se'', ``.no'' or ``.uk'' in Europe,
``.au'' in Australia, ``.net'' in North America, and ``.jp'' in Asia.

\section{DOCUMENT CREDITS}

B.T. provided the bulk of the written material; K.E. and
(to a lesser extent) J.K provided expert editing and typesetting advice.
The \playbook\ uses material borrowed from older documentation written
by the following people (in rough order of borrowing magnitude):
\begin{quote}
Sam Mackrill (lots of bits from the old FAQ) \\
Laurent Wacrenier \\
Kjetil T. Homme \\
Lars H. B. Olafsen \\
Peter Mardal \\
Rupert G. Goldie \\
Frank T. Johansen
\end{quote}
If I left anyone out, please send me some email! \\

\noindent{Brian Thomas} \\ {\tt thomas@astro.psu.edu} \\


\chapter{About Characters\index{characters}}

\section{Character Attributes\index{characters, attributes}}\label{sec:char_attr}
 
Each player interacts in the \cf\ world through the persona of a 
character. In turn, the interaction between the character and 
the game world is mediated by the ``attributes'' of the character. 
After all, the player is not going to play him/herself! In \cf\ we 
chose to parameterize the acceptable limits of player behavior 
via the values of the character attributes which, in turn, help
to determine the success of any action taken by the player. There 
are no limits per se on what you can attempt to 
do with your character; rather, the attributes of a character 
indicate the certain ``natural'' talents and inclinations.  
Two concepts which are relevant to \cf\ character attributes are 
``stats'' and ``class''.

\subsection{Character Stats\index{stats}\index{characters, stats}}

Character statistics (or ``stats'' for short) can be 
divided into two types: primary and secondary. What's the difference 
between them? Secondary stats are calculated from a number of things
including the primary stats. But the reverse isn't true, secondary 
stats never have any influence on the primary stats. In playing the game, the 
player may find that either of these kinds of stats may be changed 
for better or worse. In general, the primary stats change much less 
often than the secondary stats. Equipment, magic, and death are 
just three examples of the many things which can alter the values of 
the character stats. The current values of both the primary and 
secondary stats may be viewed in the stat window. Four important
secondary stats\emdash food, grace, hitpoints and mana also appear 
again in the stat-bar window. 


\subsubsection{Primary stats\index{stats, primary}}
\index{STR}\index{CON}\index{DEX}\index{INT}\index{WIS}\index{CHA}
\index{POW}
\index{strength}\index{dexterity}\index{constitution}
\index{power}\index{intelligence}\index{wisdom}
\index{charisma} 

The seven primary stats are:
{\small
\begin{quote} 
$\bullet$ {\bf Strength} (``STR'') $-$ 
a measure of the physical strength. \\
$\bullet$ {\bf Dexterity} (``DEX'') $-$
measures physical agility and speed. \\
$\bullet$ {\bf Constitution} (``CON'') $-$
measures physical health and toughness. \\
$\bullet$ {\bf Intelligence} (``INT'') $-$ 
measures ability to learn skills and \incantation s. \\ 
$\bullet$ {\bf Wisdom} (``WIS'') $-$ 
measures the ability to learn/use \divinemagic . \\
$\bullet$ {\bf Power} (``POW'') $-$ 
measures magical/spiritual potential. \\
$\bullet$ {\bf Charisma} (``CHA'') $-$ 
measures social and leadership abilities. 
\end{quote}
}
Primary stats have a ``natural'' range 
between 0 and $\sim$20. The actual upper limit on each primary stat is 
set by the chosen character class (see section \ref{sec:char_cls}). You 
can raise your primary stats by drinking potions\index{potions} up to 
your class natural limit.

There are plenty of items which give you bonuses to your 
stats even {\em beyond} your class limit $-$ swords\index{weapons}, 
armour\index{armour} and rings\index{rings} to 
name the most important. You can also read scrolls\index{scrolls} or cast 
spells\index{spells} (\incantation s
or prayers) to temporarily raise your stats.  The ultimate maximum value is 30, 
and the player class doesn't matter here. 

Some quantitative effects of the primary stats are summarized in table 
\ref{tab:pri_eff}. The top row in the column header specifies a particular
bonus (i.e. {\tt HpB}, {\tt MgB}, {\tt AcB}, {\tt DmB}, etc.) while the second row 
in the header indicates the stat which is used to calculate the value of 
that column. Where no stat appears, one of {\em several} stats may be 
used to calculate that value. See later parts of the text (particularly
section \ref{sec:stat_calc}) for further details. 

\begin{table}
\begin{center}
\scriptsize
\caption{Selected primary stat bonuses/penalties. \label{tab:pri_eff}}
\index{stats, limits}\index{stats, primary}
\vskip 12pt
\begin{tabular}{|r|r|r|r|r|r|r|r|r|r|}\hline 
\snglc{Stat}&\snglc{HpB}&\snglc{MgB\tablenotemark{1}}&\snglc{AcB}&\snglc{DmB}&\snglc{Thaco}&\snglc{Max Carry}&\snglc{Speed}&\snglc{\%learn\tablenotemark{2}}&\sngLc{Buy/Sell} \\ 
\snglc{ }&\snglc{(CON)}&\snglc{}&\snglc{(DEX)}&\snglc{(STR)}&\snglc{(STR)}&\snglc{(STR)}&\snglc{(DEX)}&\snglc{}&\sngLc{(CHA)} \\ \hline\hline
\input{bonus}
\hline
\end{tabular}
\end{center}
\tablenotetext{1}{Either POW or WIS can be used to calculate MgB (magic bonus).}
\tablenotetext{2}{Percentage for learning either skills (INT), \incantation s (INT) or prayers (WIS).}
\end{table}


\subsubsection{Secondary stats\index{stats, secondary}}

The secondary stats are : \\

$\bullet$ {\bf\tt score}\index{score} \emdash\ The total accumulated 
experience\index{experience} of the character. {\tt score} is increased 
as a reward for appropriate player action and may decrease as a 
result of a magical attack or character death (see section 
\ref{sec:death} for more about death). The {\tt score} starts at a value 
of 0. \\

$\bullet$ {\bf\tt level}\index{level, overall}\index{level} \emdash\ 
A rating of overall ability whose value is determined from the {\tt score}.
As the {\tt level} of the character increases, 
the character becomes able to succeed at more difficult tasks. {\tt level} 
starts at a value of 0 and may range up beyond 100. The value of the 
stat which appears in the stat window is sometimes known as the {\em overall}
{\tt level}. See section \ref{sec:experience} for more details. \\ 

$\bullet$ {\bf hit points} (``{\tt Hp}'')\index{stats, Hp}
\index{hit points} \emdash\ Measures of how much 
damage the player can take before dying. Hit points are determined from 
the player {\tt level} and are influenced by the value of the character CON
(see section \ref{sec:stat_calc}).  {\tt Hp} value may range between 1 
to beyond 500 and higher values indicate a greater ability to 
withstand punishment. \\ 

$\bullet$ {\bf mana} (``{\tt Sp}'')\index{stats, mana}
\index{mana}\index{Sp} \emdash\ Measures of how much ``fuel'' the player
has for casting \incantation s.  Mana is calculated from the character 
{\tt level}
and the value of the character POW (see section \ref{sec:stat_calc}). 
Mana values can range between 1 to beyond 
500. Higher values indicate greater amounts of mana. \\ 

$\bullet$ {\bf grace} (``{\tt Gr}'')\index{Gr}\index{stats, grace}
\index{grace} \emdash\
How favored the character is by the gods. In game terms, how much 
\divinemagic\ a character can cast. Character {\tt level}, WIS and POW effect 
what the value of grace is (see section
\ref{sec:stat_calc}).  During play, grace values 
{\em may} exceed the character maximum; grace can take on large positive
and negative values. Positive values indicate favor by the gods.\\

$\bullet$ {\bf weapon class}\index{weapon class} (``{\tt Wc}'')
\index{stats, Wc}\index{Wc} \emdash\ 
How skilled the characters melee/missile attack is. Lower values indicate a 
more potent, skilled attack. Current weapon, user {\tt level} and STR are some 
things which effect the value of {\tt Wc}. The value of {\tt Wc} may 
range between 25 and -72. 
See section \ref{sec:stat_calc} for a more detailed explanation of weapon 
class. See section \ref{sec:combat} to see how {\tt Wc} works in attacking.\\ 

$\bullet$ {\bf damage}\index{damage} (``{\tt Dam}'')\index{stats, Dam}
\index{Dam} \emdash\ How much 
damage a melee/missile attack by the character will inflict. Higher values 
indicate a greater amount of damage will be inflicted with each attack. 
See section \ref{sec:stat_calc} for a calculation of the character {\tt Dam}. 
\\ 

$\bullet$ {\bf armour class}\index{stats, Ac}\index{armour class} 
(``{\tt Ac}'')\index{Ac} \emdash\ How 
protected the character is from being hit by any attack. Lower values 
are better. {\tt Ac} is based on the character class (table \ref{tab:char_cls}) 
and is modified by the DEX ({\tt AcB} column in table \ref{tab:pri_eff}) and 
current armour worn. For characters that cannot wear armour, {\tt Ac} improves
as their level increases (see section \ref{sec:stat_calc}). \\ 

$\bullet$ {\bf armour}\index{armour} (``{\tt Arm}'')\index{stats, Arm} 
\emdash\ 
How much damage will be subtracted from successful hits made upon 
the character. This value ranges between 0 to 99\%. Current armour worn 
primarily determines {\tt Arm} value. \\ 

$\bullet$ {\bf\tt speed}\index{stats, speed}\index{speed} \emdash\
How fast the player may move. 
The value of {\tt speed} may range between nearly 0 (``very slow'') to 
higher than 5 
(``lightning fast''). Base {\tt speed} is determined from the DEX and modified 
downward proportionally by the amount of weight carried which {\em exceeds} the 
{\tt Max Carry} limit (table \ref{tab:pri_eff}). The armour worn also sets the 
upper limit on {\tt speed} (see the \spoiler\ for these limits). \\

$\bullet$ {\bf weapon speed}\index{stats, weapon speed}\index{weapon speed} 
\emdash\ Appears in 
parentheses after the {\tt speed} in the stat window. This quantity is
how many attacks the player may make per unit of time. 
Higher values indicate faster attack speed. Current weapon and user 
DEX effect the value of weapon speed. See section \ref{sec:stat_calc} 
for a calculation of weapon speed. \\

$\bullet$ {\bf\tt food} $-$ How full the character's stomach is. 
Ranges between 0 (starving) and 999 (satiated).  At a value of 0 the 
character begins to die. Some magic can speed up or slow down the 
character digestion. Healing wounds will speed up digestion too. \\
 

\subsection{Character Classes}\index{characters, classes}\label{sec:char_cls} 

Much like the older ``paper and pencil'' role-playing games 
\cf\ has adopted the idea of character ``class''. 
Each class is meant to be a template of a particular ``style'' of play; 
therefore each choice of class modifies both the starting 
values and sets the natural upper limit on the primary stats. 
{\em Important note:} character class is chosen at the time a 
character is created and can't be changed later on.
\begin{table}
\begin{center}
\scriptsize
\caption{\cf\ character classes. \label{tab:char_cls}}
\index{characters, classes}
\vskip 12pt
\begin{tabular}{|c|c|l|l|l|l|l|l|l|p{4cm}|}
\hline
Type& &         Str&    Dex&    Con&    Int&    Wis&    Pow& Cha & Special\\
\hline
\hline
\input char.tex
\hline
\end{tabular}
\end{center}
\end{table}

Table \ref{tab:char_cls} shows the various available character classes with 
the natural stat limits for each. 
Under the ``special'' column several bits of information are included. ``{\tt Ac}''
\index{Ac}\index{stats, Ac} indicates the base armour class 
for the character; ``damage''\index{stats, Dam}
indicates the base {\tt Dam}\index{Dam} value. 
Some character classes have special attack abilities and certain vulnerabilities, 
protections from, and immunities to various attacktypes. Read section 
\ref{sec:combat} for more information about what effect these can have.

\subsubsection{Sizing up the character classes}

Generally, the titles of the character classes speak for themselves. But 
you can get a better idea of the potential of a class by checking out a
few things. Take a look at the starting equipment 
(table \ref{tab:start_equip}) and, if you are using \cf\ compiled with 
the skills system, checkout the starting skills for those classes your interested 
in (table \ref{tab:skill_start}).


\begin{table}
\begin{center}
\scriptsize
\caption{Starting equipment by character class. \label{tab:start_equip}}
\index{equipment, starting}
\vskip 12pt
\begin{tabular}{|l|l|} \hline
Type & Starting Equipment \\ \hline\hline
\input equip.tex
\hline
\end{tabular}
\end{center}
\end{table}

\begin{table}
\footnotesize
\caption{Starting skills} \label{tab:skill_start}\index{skills, starting}
\vskip 12pt
\begin{center}
\begin{tabular}{|l|l|} \hline
Character class 	& Additional skills\tablenotemark{1} \\ \hline\hline
\input{skills}
\hline
\end{tabular}
\tablenotetext{1}{All character classes start with the skills {\tt melee weapons}, 
{\tt find traps}, {\tt use magic item}, {\tt literacy}, and {\tt disarm traps}.}
\end{center}
\end{table} 

Finally, here are some notes concerning a couple of the more ``exotic'' classes:\\ 
 
\noindent{$\bullet$} {\bf Fireborn}\index{Fireborn} \\ 
{\sl Attacks:} fire, physical \\ 
{\sl Protections:} immune: fire, poison; vulnerable: ghosthit, drain, cold \\ 
{\sl Special:} fly, no armour, no weapons. {\tt Ac} 0 \\ 
Fireborns are fire spirits. They're closely in tune with
magic and learn all types of magic easily. 
Being fire spirits, they are
immune to fire and poison, and vulnerable to cold. They are vulnerable to
ghosthit and drain because being mostly non-physical, anything which strikes
directly at the spirit hits them harder. \\ 

\noindent{$\bullet$} {\bf Monk}\index{Monk} \\ 
{\sl Attacks:} physical \\ 
{\sl Protections:} none \\ 
{\sl Special:} no weapons. \\ 
Monks are members of various martial arts orders. They have devoted themselves to
a life of contemplation and revelation though physical fighting! Their
life-long secret vows keep them from using all hand-held melee weapons, 
but in return they are allowed to learn secret techniques of meditation. \\ 

\noindent{$\bullet$} {\bf Quetzalcoatl}\index{Quetzalcoatl} \\ 
{\sl Attacks:} physical \\ 
{\sl Protections:} immune: fire; vulnerable: paralyze, poison, cold \\ 
{\sl Special:} no armour, {\tt Ac} 5 \\ 
Quetzalcoatls are an odd mixture of magic and combat abilities. They are
born knowing the spell of burning hands (heh, its their ``dragon breath''). 
But because of their low natural intelligence/wisdom, they have a very hard 
time learning new spells. All the same, they may become potent 
wizards/priests as they have the highest power bonus, and they will 
typically have a large amount of mana and a fair amount of grace. 
Quetzalcoatls can be very
devastating fighters at low level. A combination of  their low natural {\tt Ac} 
and high base damage tend to make mincemeat
out of low-level monsters. However, at mid-level, they really begin to have
problems because they cannot use armour. \\ 
 
\noindent{$\bullet$} {\bf Wraith}\index{Wraith} \\ 
{\sl Attacks:} cold, physical \\ 
{\sl Protections:} immune: drain, ghosthit; protected: physical, cold; vulnerable: fire \\ 
{\sl Special:} {\tt Ac} 6 \\ 
The Wraith is a creature of the undead. These characters represent revenging spirits come
back to life to work their unholy will on the living. Their undead nature 
makes them immune to life-damaging forces and their etherealness protects
them from physical harm. Like all undead, they succumb to fire readily. 

\section{Generating a character}\index{characters, generation}\label{sec:char_gen}

When you start up \cf , you will be asked for a character name
followed by a password. If you are playing for the first time, type
any name you like; this will be your character name for the rest of
the life of that character. Next, type in any password. 
{\em Remember!:} 
you will need to type the same password again to play that character
again! One more note: its not a good idea to use ``real'' passwords 
to your computer accounts! Doing so may make your system vulnerable
to unscrupulous \cf\ server administrators. 

Your next step will be to generate random (primary) stats for your 
character. 
You aren't limited to the number of times you can roll your stats\emdash 
so have fun. 
Notice that the stats are always arranged that the STR stat has 
the highest value,
the CHA stat has the lowest. You may re-arrange the order of these values
when you decide that you have rolled some decent stats. 
{\em Two points:} you can 
never roll a character with better stats than an average of straight 15's, 
and you can't roll higher than 18 in a stat. 

When you roll your character, the stats displayed are the stats you 
will get as a human (which are unmodified). 
When satisfied, you can step through a number 
of classes, each with special bonuses in stats.

Table \ref{tab:char_cls} shows how your basic stats will
be changed by choosing a different class.
The difference between the natural stat limit and 20 indicates the 
bonus/penalty assigned to rolled primary stats. For example, a 
barbarian has a maximum strength which is 4 higher than 20$-$that 
means he will begin with an additional 4 points added to his
strength roll. On the other hand, a barbarian can never get above 14
in intelligence.  This means that your rolled character will have 6
less in intelligence if you choose that class.  It also means that you
can't be a barbarian if you roll less than 7 in intelligence; the
poor barbarian would have had a stat below 1.
 
\subsubsection{Selecting a character}

While each class has its particular strengths and weaknesses, in summary
its just a fact that some classes are easier to play than others.

For beginning players, the ``simple fighter'' characters are the easiest to
play successfully. The Dwarf, Human, and Warrior are among good earlier 
choices. As you gain experience
with using \cf\ you may wish to branch out into other ``fighter'' characters
such as the Thief or Elf, or try your hand at playing ``spellcaster'' 
characters like the Wizard, Mage, Cleric or Priest. 
The ``exotic'' classes (e.g. Fireborn, Monk, Quetzalcoatl, and Wraith) are 
the hardest classes to master. 

One more note: If you are using the default game (compiled with the skills 
system), 
you will probably find any character class that has a low natural INT (for 
example, the Barbarian class) will have a fairly difficult time at higher 
levels. Skills, and most importantly, spells, will be more difficult
learn. Plan to spend a lot of loot on spell books (\wizbook s and prayer books)
and skill scrolls.


\chapter{Playing \cf}

\section{Basics}\label{sec:basic}

In this section, several basic bits of information are detailed in
a concise way in rough order of importance.
Various pointers to other sections of this document will help you to
round out your knowledge if you want to. All of the available player
commands are concisely explained in appendix \ref{app:commands}. You
can always get a summary of available commands while playing the game;
hit ``{\tt ?}'' for help. \\

\subsection{How to do simple stuff} \index{How to do simple stuff}

\subsubsection{Move around and attack}\index{commands, movement}\index{movement}\index{combat}\index{running}
Movement is accomplished with the mouse, or
with the same keys that some rouge-like computer games use. To move using
the mouse, position the cursor over a square you wish to move to
in the view window
then click the right hand button. If you want to use the keys, here's a
simple diagram of where the various movement keys will take you: \\
\begin{center}
\begin{tabular}{ccccc}
{\tt y} &  & {\tt k} &  & {\tt u} \\
  & $\nwarrow$  & $\uparrow$  & $\nearrow$ &   \\
{\tt h} & $\longleftarrow$ & .  & $\longrightarrow$ & {\tt l} \\
  & $\swarrow$  & $\downarrow$  & $\searrow$ &   \\
{\tt b} &  & {\tt j} &  & {\tt n} \\
\end{tabular}
\end{center}
The ``{\tt .}'' refers to yourself; you don't move anywhere when you
press it.
In order to ``run'' in a particular direction (i.e. move continuously
without having to repeatedly punch the key) hold down the control
key then hit any movement key or the right hand mouse button to
start moving. Release the {\tt $<$control$>$} key when you wish to stop running.

If you move into something, you will attack it. This means walls,
doors, and monsters will be damaged if you hit them. Players and
friendly monsters may also be attacked in this way, but only if
you set the peaceful flag to ``off''. To learn more about the combat
system see section \ref{sec:combat}. \\

\subsubsection{Pickup stuff}\index{commands, pickup}\index{picking up items}
To pickup items, move over the item, then either hit the ``{\tt ,}'' key
or move the cursor over to the look window, position it over the desired
item and click the left mouse button. You will see the item appear in your
inventory window. If you pick up too much stuff, you won't be
able to see it all at once. Use the ``{\tt $<$}'' and ``{\tt $>$}'' keys
to rotate through the inventory list. \\

\subsubsection{Applying stuff: wear armour, wield a weapon, eat, and so on.}\index{commands, apply}
Most of the time, in order to manipulate or ``{\tt apply}'' items you have
to be holding them. Move the cursor over to the desired item in the
inventory window. By using the middle button on the mouse, you may
toggle the status (ie between ``applied'' or ``unapplied'') of items.
Note that
some items when applied will be used up (they disappear from the
inventory window). Examples of these kind of
items include ``food''\inputimage{foodone}, ``potions''\inputimage{potion},
and ``scrolls''\inputimage{scrolls}.
To learn more
about the uses of various items see chapter \ref{chap:equip}. \\

\subsubsection{Shoot a ranged weapon}\index{commands, fire}\index{bows}
\index{wands}\index{rods}\index{horns}
Ranged weapons include bows\inputimage{bows}, wands\inputimage{wands},
rods\inputimage{rods}, or horns\inputimage{horns}. Apply the desired
weapon, then check to see that the {\tt Range:} slot in the status window
indicates that item is ``readied'' (yes...you can have something applied but
not readied). If its not ready, use either the plus or minus keys to
rotate though all of the slots. Once readied, use the ``{\tt $<$shift$>$}'' key
followed by a movement key to fire the object in that direction. Alternatively,
place the cursor in the view window, then hit the middle mouse button to fire. \\

\subsubsection{Enter a building or boat.}\index{commands, apply}
\index{entering buildings}\index{movement}\inputimage{guild}
Move over on top of the desired structure. Then hit either middle mouse
button while the cursor is on the icon of the structure in the look window,
or hit the {\tt A} key. If there is a link to a map drawn of the ``inside''
you will be taken there. If no link exists, you will get the message
``{\tt the $<$structure$>$ is closed.}''. \\

\subsubsection{Use a skill}\index{commands, ready\_skill}
\index{skills, how to use}
In order to use a skill, it must first be readied. You can ready any skill
you have with the {\tt ready\_skill} command. Also, some skills will
auto-matically be readied when you undertake certain
actions. For example, if you run into a hostile monster with a wielded weapon
the {\tt melee weapons} skill is readied. A ready skill will show up in the
stat window in the {\tt Range:} slot. If a skill doesn't appear in the slot, rotate
the range slot to check for the skill. When a skill is readied, the range slot will
appear as ``{\tt Skill: $<$skill$>$}'' (otherwise it appears as
``{\tt Skill: none}'').
To use the skill, make a ``ranged attack'' (ie hit the same keys or
mouse button as for firing a wand). To learn more about the skills
system see chapter \ref{chap:skills}. \\

\subsubsection{Cast a spell}\index{spells, how to use}\index{commands, cast}
\index{talisman}\index{holy symbol}
In order to ``{\tt cast}'' spells (either \incantation s or prayers), you must have
the skills of {\tt \spellcasting\ } (\incantation s) or {\tt praying} (prayers).
Possession
of a ``talisman''\inputimage{talisman}\ or a ``holy symbol''\inputimage{holysymbol}\
will also allow you to respectively {\tt cast} \incantation s or prayers). You can
only {\tt cast} those spells you have {\em learned}. Issue the meta-command
{\tt `cast $<$spell$>$} to ready a spell in the {\tt Range:} slot. To
``fire'' the spell, make
a ranged attack. Note! if you don't have enough mana a\ina\ \incantation\
{\em} will fail.
If you don't have enough grace a prayer {\em may} work. For more information
concerning the magic system see chapter \ref{chap:magic}.

\subsection{Saving the game and ending the \cf\ session:}\index{commands, quit}
\index{commands, save}\index{quitting}\index{saving}

The {\tt save} command is to provide an emergency backup in case of a game crash.
To save your player at the end of your game session you must find a ``Bed to
reality''\inputimage{savebed}, move your player over it and
{\tt apply} it (``{\tt A}''). These beds can usually be
found in the inns and
taverns dotted around the maps (especially in cities). This prevents you
just saving anywhere and forces you to finish what you are doing and return
somewhere safe.

\subsection{About NPCs}\index{commands, say}\index{commands, \"}
\index{NPC}\index{talking}
{\em N}on {\em P}layer {\em C}haracters are special
``monsters'' which have ``intelligence''. Players
may be able to interact with these monsters to help solve puzzles and find
items of interest. To speak with a monster you suspect to be a NPC, simply
move to an adjacent square to them and push the double-quote, ie. {\tt "}. Enter
your message, and press {\tt $<$return$>$}. You can also use the meta-command
{\tt 'say} if you feel like typing a little extra.

Other NPCs may not speak to you, but display intelligence with their
movement. Some monsters can be friendly, and may attack the nearest of your
enemies. Others can be ``pets'', in that they follow you around and help you
in your quest to kill enemies and find treasure.


\subsection{Important special items}

There are many special items which can be found in \cf , of those shown
below, the most important may be the signs. \\

$\bullet$ {\bf Signs:}\index{signs}\inputimage{sign}

Signs often have messages that might clue you in on quests and puzzles
or even refer you to NPCs. It is your job as a player to make sure you
read every sign to make sure you do not miss things. \\

$\bullet$ {\bf Handles and Buttons:}\index{handles}\index{buttons}
\inputimage{handbutt}

These items can often be manipulated to open up new areas of the map.
In the case of handles, all a player must do is apply the handle. In
the case of buttons, the player must move items over the button to
"hold" it down. Some of the larger buttons may need very large items to
be moved onto them, before they can be activated. \\

$\bullet$ {\bf Gates and locked doors:} \inputimage{gatedoor}
\index{doors}\index{gates}

Gates are often tied to a handle or button and can only be opened by
manipulating the the handle/button. Doors that are locked can either be
smashed down by attacking the door, by using keys\inputimage{keys}\
which can be found
throughout the game, or by picking the lock. \\

$\bullet$ {\bf Pits:}\inputimage{pit}
\index{pits}

Pits can be doorways to new areas of the map too, but be careful, for
you could fall down into a pit full of ghosts or dragons and not be
able to get back out! \\

$\bullet$ {\bf Break away walls:}\inputimage{br_wall}
\index{walls}

Are a common occurrence in \cf . These type of walls can be
"destroyed" by attacking them. Thus, sometimes it may be worth a
player's time to test the walls of a map for ``secret doors''. \\

$\bullet$ {\bf Fire walls:} \inputimage{fr_wall}

Will shoot missiles (including bullets, lightning, etc.) at players.
Some firewalls can be destroyed while others cannot. \\

$\bullet$ {\bf Spinners and Directors:} \inputimage{dir_spin}
\index{spinners}\index{directors}

These odd items will change the direction of any item flying over them,
such as missile weapons and spells. \\


\section{Matters of life and death}

\subsection{Attack system}
\label{sec:combat}
\index{combat}\index{attacktypes}\index{slaying}\index{Dam}\index{Wc}\index{Ac}

Every time you make an attack, your attack is classified with one or more
``attacktypes''. For example, an attack made with a ordinary sword
results in the attack being made with the attacktype of ``physical''.
For another example, if a Mage attacks with a fireball spell the
attack is made with the ``magic'' and ``fire'' attacktypes. In
similar fashion, a defender may be protected, vulnerable, or immune
to any attacktype. \\

\noindent{Here} is a summary of the attack system; in fact, its a
bit more complicated. \\

\subsubsection{Hitting an opponent}
\indent{Several} quantities are involved in determining whether an attack will
hit its target.  The attacker will hit if his {\tt Wc} is less than or equal
to the
defender's {\tt Ac} + 1D20\footnote{a random number between 1 and 20} or
if the D20 gives a 20 (remember, both {\tt Wc} and {\tt Ac} improve as
their value drops {\em lower}).\\

\subsubsection{Damaging a hit opponent}
\indent{Damages} are randomly generated, with the magnitude of the random number
being based on the attacker's {\tt Dam} stat.
If defender is immune to an attacktype
in the attack, he receives no damage, if he's protected he receives
half damage, if he's vulnerable to this kind of
attacktype, he receives double damage. For physical attacks, the percent
of your {\tt Arm} value is subtracted from incoming damage.
Some magic weapons can ``slay'' various races of creatures. If the weapon
slays the defender, the attack damage is tripled.


\subsection{Experience}\label{sec:experience}\index{experience}
\index{experience, categories}\index{level, overall}\index{level}

Accumulation of experience will result in increasing the {\tt level} of the
player's character. In a \cf\ game where skills are {\em not} present,
experience is only gained for removing traps and killing monsters.
A player will gain a new {\tt level} when their
experience total reaches a new amount in the hierarchy shown
in table \ref{tab:exp_level}.

In the skills-based game, {\em several} kinds of experience exist. The
{\tt score} represents the ``{\em overall}'' proficiency of the player
and is the
{\em sum} of all the differing kinds of experience possessed.

The player accumulates experience into various ``{\em experience
categories}'' according to their actions.
Another way of putting this is that you become better
at what you do most often and most successfully. For example,
a player who kills monsters with \incantation s gains experience in
casting \incantation s. A player who steals from creatures often will
get better at stealing, and so on.

Each experience category will have a {\tt level} assigned to it based on
the amount of experience accumulated in it (using the same schedule
 shown in table
\ref{tab:exp_level}). Use the {\tt skills} command to
investigate which skills your character has and to see the {\tt level} of
ability you have in each experience category. See chapter
\ref{chap:skills} for more information about skills and skill-based
experience.

As an aside, monsters {\em also} are assigned a {\tt level} of proficiency
and may gain experience. The main way which monsters gain experience
is by {\em killing} players! Beware going after a monster that has
killed several players, it will be much more difficult to challenge!

\begin{table}
\begin{center}
\caption{Relationship between experience and {\tt level} for the first
10 levels. \label{tab:exp_level}}
\vskip 12pt
\small
\begin{tabular}{|rl|}
\hline
{\tt Level}	& \multicolumn{1}{c|}{Experience} \\ \hline\hline
\input{levels}
... & ... \\
\hline
\end{tabular}
\end{center}
\end{table}

\subsection{Calculation of selected secondary stats}\label{sec:stat_calc}
\index{stats, secondary}

Both the primary stats and {\tt level} of the character will influence
the secondary stats given below. In the skills-based game, the
appropriate experience category\footnote{categories
are given for the default settings, this can be changed by the
server administrator} is identified for purposes of determining
which {\tt level} is used in the calculation. In the non-skills
game, the {\em overall} {\tt level}\index{level, overall} is always used.

Refer to table \ref{tab:pri_eff} if a quantity in a calculation is
left unexplained.

\subsubsection{Weapon class ({\tt Wc})}\index{Wc}\index{stats, Wc}
The {\tt Wc} calculation is:
\begin{quote}
class {\tt Wc} - {\tt thaco}(STR)\index{thaco} - weapon {\tt Wc} - {\tt level} - 1 every 6 {\tt level}s
\end{quote}
where {\tt thaco} is found using the STR stat on table \ref{tab:pri_eff},
the weapon {\tt Wc}
can be determined from the \spoiler\ and the {\tt level} is taken from the
{\tt physique} experience category.

\subsubsection{Weapon Speed}\index{stats, weapon speed}
\index{weapon speed}
The calculation for weapon speed is quite complex (but here it is!).
The value for the weapon speed is:
\begin{quote}
{\tt speed}/({\tt LF} * {\sl NastyFactor} * {\sl LessNastyFactor})
\end{quote}
where {\tt LF} (``level factor'') is a number between 0.8 and 1.2 that
increases with the {\tt level} as (4+{\tt level})/(6+{\tt level})*1.2,
(the {\tt level} is taken from the overall {\tt level}),
and both {\sl NastyFactor} and {\sl LessNastyFactor} are calculated below.

The value of {\sl NastyFactor} is:
{\small
\begin{quote}
2/3 + {\tt MaxC}/363 - ((0.00167*{\tt WpnW})/{\tt MaxC}) + ({\tt speed}/5) + ((Dex-14)/28)
\end{quote}
}
\noindent{where} {\tt MaxC} is the maximum carrying
limit of the character, {\tt WpnW} is
the weapon weight. Weapon weight can be determined from examining the given
weapon (ie look at the number to the right side of the weapon icon in the
inventory window).

The value of {\sl LessNastyFactor} is:
\begin{quote}
2 - ((WpnF - {\tt magic}/2)/10)
\end{quote}
where {\tt WpnF} is the ``weapon factor'' and {\tt magic} is the
value of the enchantment on the weapon (i.e. $+$1, $+$2, etc.).
Weapon factor\index{weapon factor} is given for some weapons
in table \ref{tab:weap_factor}. \\

\begin{table}
\begin{center}
\caption{Weapon factor ({\tt WpnF}) for selected weapons}\label{tab:weap_factor}
\small
\vskip 12pt
\begin{tabular}{|crlc|} \hline
	& & & \\
       &  1.6 & nunchacu &  \\
       &  1.5 & dagger & \\
       &  1.4 & Sting, katana, shortsword, taifu & \\
       &  1.3 & Belzebub's sword, Darkblade, Excalibur & \\
       &  1.2 & Firebrand, Frostbrand, Mjoellnir, Mournblade, &  \\
       &      & Stormbringer, broadsword, falchion, light sword & \\
       &      & long sword, quarterstaff, sabre, sword, trident & \\
       &  1.1 & Deathbringer, Demonbane, Dragonslayer, Holy Avenger & \\
       &      & club, hammer, mace, unicorn horn & \\
       &  1.0 & axe, stonehammer & \\
       &  0.9 & Skullcleaver, morningstar, stake, stoneaxe & \\
       &  0.8 & large morningstar & \\
       &  0.5 & Bonecrusher, Gram, shovel & \\
       &  0.3 & large club &  \\
       &  0.1-0.4 & chair & \\
       &  0.1 & bed & \\
	& & & \\ \hline
\end{tabular}
\end{center}
\end{table}


\subsubsection{Damage ({\tt Dam})}\index{stats, Dam}\index{damage}

Current weapon, character class
(table \ref{tab:char_cls}) and STR ({\tt DmB} in table \ref{tab:pri_eff})
all effect the value of {\tt Dam}. The calculation for {\tt Dam} is:
\begin{quote}
Class bonus $+$ {\tt DmB} $+$ Weapon damage $+$ Skill damage
\end{quote}
In addition, for every 4 levels of fighting expertise ({\tt physique}
experience category level) 1$+$({\tt Dmb}/5)) is added to {\tt Dam}.
Some skills (namely the hand-to-hand, or martial arts skills)
can add significant damage to the overall total. The amount
that is added depends on the user's {\tt level} in that skill.
Note that weapon damage is automatically added to your damage
rating in the stat window when you wield any weapon.


\subsubsection{Armour class ({\tt Ac})}\index{stats, Ac}\index{Ac}

For characters that cannot wear armour (ex. Fireborn), their
{\tt Ac} can improve as their score increases. The calculation for {\tt Ac} is then:
\begin{quote}
Class {\tt Ac} $-$ {\tt level}/3
\end{quote}
The value of the character {\tt Ac} may decrease to the value of -10.
After that, no more improvement is made regardless of earned
experience. The overall {\tt level}\index{level, overall} is
{\em always} used for this calculation.


\subsubsection{Hit points ({\tt Hp})}\index{hit points}\index{stats, Hp}

A character will gain the following number of {\tt Hp} for {\em each} new
overall {\tt level} gained between levels 1$-$10:
\begin{quote}
({\tt HpB})/2 + D4 + D4 + 1 + 1/every even {\tt level} {\em if} {\tt HpB} is even.
\end{quote}
Where D4 is a random number between 1 and 4 and you always get 1 hit point
gained per {\tt level}, regardless of {\tt HpB}.
For levels after 10th, {\tt Hp} grow 2 per {\tt level}.
The overall {\tt level} provides the correct value for {\tt level} in
the calculation.


\subsubsection{Mana ({\tt Sp})}\index{mana}\index{stats, mana}

A character will gain the following number of {\tt Sp} for {\em each} new
{\tt magic} experience category {\tt level} gained between levels 1$-$10:
\begin{quote}
({\tt MgB}(POW))/2 + D3 + D3 + 1 every even {\tt level} {\em if} {\tt MgB} is even.
\end{quote}
Where D3 is a random number between 1 and 3 and you always get 1 mana gained
per new {\tt level}. For {\tt magic} levels past the 10th, mana grows 2 per level.


\subsubsection{Grace ({\tt Gr})}\index{stats, grace}\index{grace}

A character will gain the following number of {\tt Gr} for {\em each} new
{\tt wisdom} experience category {\tt level} gained between levels 1$-$10:
\begin{quote}
({\tt MgB}(WIS) + {\tt MgB}(POW))/8 + D3
\end{quote}
Where D3 is a random number between 1 and 3 and you always get 1 grace gained
per new {\tt level}. For levels past the 10th,
grace grows 1 per {\tt wisdom} {\tt level}.

\subsection{Death $---$ the ultimate penalty?}\label{sec:death}

Should your hit point total drop below 0 you will die\inputimage{gravestone}.
What happens next depends on the options the server administrator choose.

If the {\tt NOT\_PERMDEATH} option is being used (this is the default) then your character
 (and his all of his {\em carried}
equipment at the instant of death) will appear at the last savebed you used.

For each time you die, your character loses 20\% (this is the default and can be different
depending on the server).
of their experience (in all categories) and a random primary stat decreases by 1.
At low levels, the stat loss is bad news, while it is the experience loss that
{\em really} hurts at those high levels. You may still re-gain lost stats (up to your
natural limit) by drinking potions.

If the server administrator has made death permanent, there is still hope
for your character. When the {\tt RESURRECTION} option is enabled (the default option
when {\tt NOT\_PERMDEATH} is {\em not} being used) other characters may bring you back
from the beyond. There are 3 spells that are capable of doing this, but
you may need to retrieve the corpse\inputimage{corpse}\ of the character
that is to be resurrected!


\section{Some advanced stuff}

\subsection{Some useful advanced commands}

\subsubsection{Meta-command ({\tt '}) }\index{meta-command}\index{commands, meta}
Any command listed by the use of help ({\tt ?} key) can be
used by spelling it out after the meta-command is issued. For example, you
can use the command {\tt say} either by hitting the double-quote ({\tt "} key) or
by issuing the command string {\tt 'say}.

\subsubsection{Binding commands ({\tt 'bind} and {\tt 'unbind})}\index{binding}\index{unbinding}
\index{commands, bind}
\index{commands, unbind}
You may bind any key with a complex command. For example, you could
bind the use of the meditation skill to the key ``{\tt m}''. To do this first
type:
\begin{quote}
{\tt 'bind use\_skill meditation }
\end{quote}
then press {\tt $<$return$>$}. The game will then ask you for a key to bind the
command to, you then hit {\tt m}. You can also re-bind this key to something
else later if you wish too. Issue the command:
command:
\begin{quote}
{\tt 'unbind reset }
\end{quote}
to totally reset bindings on keys.

\subsubsection{Pick up toggle ({\tt @})}\index{picking up items}\index{commands, pickup}
This allows you to change your pickup status. Eight different modes
for pick up exist: ``don't pick up'',``pick up 1 item'', ``pick up 1 item and
stop'', ``stop before picking up'', ``pick up all items'', pick up all items
and stop'', ``pick up all magic items'', ``pick up all coins and gems''.
Whenever you move over a pile of stuff your pickup mode controls
if and what you collect. You can always pickup stuff using the pickup
command ({\tt ,}) regardless of your current pickup mode.

\subsubsection{Invoke ({\tt 'invoke})} \index{commands, invoke}
A useful way to quickly cast both \incantation s and prayers is via the invoke
command. To use it effectively, {\tt bind invoke $<$spell$>$} to any
key. Then, when that binded key is pressed, your character will cast that magic
in the direction they're currently facing.

\subsection{Playing with other people}

As a general rule, other \cf\ players will prefer to co-operate or
at least leave each other alone. If you go about killing other player's
characters you may not only risk their continued wrath, but the anger of
the server administrator as well. Check out the house rules before you
start slaying players.

\subsubsection{Useful multi-player commands}

Here are some useful commands for playing with other players: \\

\noindent{\bf\tt shout}\index{commands, shout} \\
This meta-command will broadcast your message to every player currently
logged in. The say command only sends messages to players who share your
current map. Invoke this command just like {\tt say}.\\

\noindent{\bf\tt  who}\index{commands, who} \\
This will give you a listing of all the current players and the maps
they are on currently. Invoke this command as {\tt 'who}.\\

\noindent{\bf\tt tell}\index{commands, tell} \\
Will send your message only to the player indicated. It is invoked as:
{\tt 'tell $<$character name$>$ $<$message string$>$}.\\

\subsubsection{The simple party system}\index{party system}\index{commands, party}
If the {\tt SIMPLE\_PARTY\_SYSTEM} has been enabled on your server you can
use this to adventure with other players. All experience gained by
members of the party is split equally, and in addition, party members
are always peaceful towards each other. Here's 2 useful party commands:\\

\noindent{1)} To form a party issue the command:
\begin{quote}
{\tt 'party form $<$party name$>$ }
\end{quote}
2) To join a party, type:
\begin{quote}
{\tt 'party join $<$party name$>$ }
\end{quote}
To see all of the options, type {\tt 'party help}.




\chapter{Magic System} \label{chap:magic} 

\section{Description}
\index{magic}\index{magic, system}\index{\wizardry\}\index{\divinemagic\}
\index{spells}

Two broad categories of magic exist in \cf : ``\wizardry '' and 
``\divinemagic ''. The fundamental difference between the two comes 
down to the source that powers the magic of each. 
In \divinemagic\ the practitioners, ``priests'', do not use their own power
but rather channel power from divine entities (``gods''). They utilize
various ``prayers\index{prayers}'' to cast their magic and grace is the measure
of how much magic the priest may channel. The higher the 
level of the priest and the better his wisdom and power, the more 
grace the priest will have in the eyes of his god.
In the practice of \wizardry\  a ``wizard'' calls upon his own lifeforce
(or ``mana'') to power his arcane \incantation s.\index{\incantation s} Mana 
is based on of the wizard's innate power but may be increased through 
his skill in \wizardry . 

The scope and sphere of these two magics are different. Through the use of 
\divinemagic\ the priest has access to powerful spells\footnote{ A ``spell'' 
is a common name referring to both prayers and \incantation s.\index{spells}} 
of protection,
healing, and of slaying {\em unholy} creatures. If the multiple gods 
option is used\footnote{this is the default}, the god a priest worships
will have other impacts on the priest's magic and abilities (see section 
\ref{sec:multigod}). 
In contrast, \wizardry\ is more oriented towards the harnessing of elemental 
forces of creation, alteration and 
destruction. There are two minor variants of \wizardry : alchemy (section
\ref{sec:alchemy}) and rune magic (section \ref{sec:rune}).

Each form of magic is orthogonal to the other. In some {\em no magic} 
areas, the wizard is blocked from accessing his store of mana, but the
priest may operate his magic normally. Similarly, there are {\em unholy} 
areas in which the priest loses his contact with his god and cannot 
cast magic; in unholy areas the wizard is unhindered. Of course, no 
magic and unholy areas can sometimes coincide. 

In addition, wizards 
have the handicap that if they are encumbered with 'stuff', they are 
less effective at \incantation s.  Heavy weapons and 
heavy armour are the main cause spell-failures. See the section
on encumbrance\index{encumbrance} (section \ref{sec:encumberance}) for details.  
Weapons and armour have no effect on the practice of \divinemagic\ 
but grace regenerates slower than mana, and the amount of grace
that a priest possesses helps to determine the success of their
prayers.  
 
\section{Learning spells}\label{sec:spell_learn}
\index{spells, how to learn}

Both types of spells may be learned by reading books (see section 
\ref{sec:items}). 
The overall chance of learning a spell uses the following formula \\ 
\begin{center}
\% chance to learn $=$ (base chance $+$ (2$\times${\tt level}))/1.5 \\ 
\end{center}
The base chance that a prayer/\incantation\ will be learnt is based on WIS/INT
respectively. Look at table \ref{tab:pri_eff} to find your {\em base}
chance in the learn\% column. If you are attempting to learn a {\em prayer},
you would use your WIS stat to find the base chance. Likewise, the {\tt level}
used in the formula is related to the type of spell. If you are attempting
to learn an \incantation , the value of level to use is your \mage\ experience
level (and you use the \priest\ experience level for learning prayers). 
Once your chance to learn a spell exceeds 100%, you always suceed in 
all attempts to learn spells. 
 

\section{Magic paths} \label{sec:magicpath} 
\index{magic, paths}

Long ago a number of archmages discovered patterns in the web that spells
weave in the aether. They found that some spells had structural similarities
to others and some of the mages took to studying particular groups of spells.
These mages found that by molding their thought patterns to match the patterns
of the spells they could better utilize all the spells of the group. Because
of their disciplined approach, the mages were described as following spell
Paths. As they attuned themselves to particular spell Paths they found that
they would become repelled from others, and in some cases found they were
denied any access to some paths. The legacy of these mages remains in some
of the magical items to be found around the world. Use of these ``attuned'' 
items will strongly effect the quality of the \incantation s and prayers cast by 
the magician. See section \ref{sec:multigod} to see how the worship of 
a god might effect the spell casting of the magician. 


\subsubsection{Technical details}

The Paths themselves are given in table \ref{tab:spath}.

A character (or NPC) that is attuned to a Path can cast \incantation s/prayers from that 
Path at 80\% of the mana/grace cost and in addition receives duration/damage 
bonuses as if the caster were five levels higher. A person that is repelled 
from a Path casts \incantation s/prayers from that Path
at 125\% of the mana/grace cost and receives duration/damage bonuses as if
the caster were five levels lower (minimum of first level). 
The casting time is also modified by 80\% and 125\% respectively. 
If a wizard or priest is denied access to a Path they cannot cast any spells from it. 

\begin{table}
\begin{center}
\caption{Known Spell Paths \label{tab:spath}}
\index{magic, paths}
\small
\vskip 12pt
\begin{tabular}{|clllllc|} \hline 
& & & & & & \\
\input{spellpath}
& & & & & & \\
\hline
\end{tabular}
\end{center}
\end{table}

Paths are quite powerful; they don't come cheaply. Most magical items
with path\_attuned attributes will have path\_repelled and path\_denied
attributes as well, to balance them out. 
% The same goes for worshiping
% gods; most gods will prevent you from casting one or more paths of magic. 
 
\section{Multiple gods} \label{sec:multigod} 
\index{cults}\index{altars}\index{gods}\index{\divinemagic\}
\index{magic, gods}

Gods in \cf\ are not omnipotent beings. Each is thought of possessing
a certain sphere of influence, indeed, some philosophers have thought
that the gods might spring from the same mystical patterns that form
the spell Paths. Certainly it appears that each of the gods embodies
one or more of these Paths (but not all of them!!). Because the gods
are not omnipotent, we often speak of their religions as being 'cults'.

Under the multigod option, priests are allowed
to select from an array of different gods. Worship of each god is unique, 
and brings differing capabilities to the priest. 
See appendix \ref{app:gods} for a listing of the gods and some of the 
attributes/effects of worshiping of these cults.


\subsection{Joining a cult}
\index{gods, worship}\index{cults}

Praying at {\em aligned} altars\inputimage{altar}\ is the usual way 
in which a priest interacts
with their god/cult. Aligned altars are identified by their name (e.g. 
altar of $<$god's name$>$) and may be found in various maps all over the 
world of \cf .
When a player prays over an aligned altar, one of three things may happen based
on the players currently worshiped god:
\begin{quote}

        (1) {\bf "Unaligned" player prays over an altar} $-$ 
	results in that player
        becoming a worshiper of the god the altar is dedicated to.

        (2) {\bf Player prays over their god's altar} $-$ 
	results in faster grace
        regeneration. In addition, player may pray to gain up to twice their
        normal amount of grace. Also, from time to time your god might
	give you information, blessings, or something really good; it depends
	on your WIS, POW and \priest\ experience.

        (3) {\bf Player prays over alien god's altar} $-$ 
	results in punishment
        of the player (generally they lose some of their \priest\ experience). 
	This action {\em can} result in the defection of the player to the alien 
	god's cult.
\end{quote}
Note that once a player has joined a cult, it is impossible to go back
to being ``unaligned'' to any god.


\subsubsection{Summary of benefits/penalties for joining}

The following things happen when a worshiper joins a god's cult:
\begin{quote}

- the worshiper gains access to the special flavor of magic belonging
to the cult (see table \ref{tab:priest_prayer}).

- the ability to cast magic is altered to reflect the powers
of the worshiper's god. Some spells will be easier to cast; others
will be more difficult, and some spell Paths will be forbidden.
It is impossible to regain forbidden spells by any means except
leaving the cult.

- the worshiper becomes protected and/or vulnerable to certain attacks. 
\end{quote}
Note that a player can belong to only {\em one} cult at any one time.

\begin{table}
\begin{center}
\footnotesize
\caption{Special priest prayers. \label{tab:priest_prayer}} \index{prayers} 
\vskip 12pt
\begin{tabular}{|p{0.5cm}llp{0.5cm}|} \hline
& Prayer & Description & \\ \hline\hline  
 & & & \\ 
    &    Bless                   &  Enhances the recipients combat ability & \\ 
    &                                 &  and confers some of the gods special & \\
    &					& sphere of protection. & \\ 
 & & & \\ 
   &     Banishment 		&  An {\tt AT\_DEATH}\tablenotemark{1} attack is made versus & \\
	& & enemies of the caster's god. &  \\
 & & & \\ 
   &     Call holy servant     &  Weaker version of an avatar is summoned.& \\
 & & & \\ 
   &     Cause wounds	    &  These prayers use the attacktype of & \\
   &                                  &  ``godpower''. This means they will effect & \\
&				     & magic immune creatures AND each prayer has & \\
&				     & the special attacktype(s) of the priest's god. & \\
 & & & \\ 
&        Consecrate            &  Dedicates an altar to the caster's god.& \\ 
 & & & \\ 
&        Curse                  &  Decreases the recipients combat ability & \\ 
&                                     &  and confers some vulnerabilities particular & \\
&                                     &  to the caster's god. & \\ 
 & & & \\ 
&        Holy orb              &  Its like a fireball, but has the same effect & \\
&			     &  as holy word\tablenotemark{2}. This prayer is most effective & \\
&			& against single creatures. & \\
 & & & \\ 
&        Holy word\tablenotemark{2} &  This prayer shoots forth a cone of power & \\
&				     &  that will damage only enemies of the caster's & \\
       &                              &  god. & \\
 & & & \\ 
       & Holy wrath           &  Currently the most powerful ``holy word''\tablenotemark{2} & \\ 
       &                              &  prayer available. & \\ 
 & & & \\ 
       & Summon avatar         &  Summons a "golem" that is tailored to & \\ 
       &                              &  the powers of the worshiped god. This & \\ 
       &                              &  prayer is more powerful (in general) & \\ 
       &                              &  than a summoned elemental and is one & \\ 
       &                              &  of the priest's most potent attack spells. & \\ 
 & & & \\ 
       & Summon cult monsters  &  	Summons creatures friendly to the priest's & \\
       &                              &  god. Depending on the god this can be a & \\ 
       &                              &   powerful or wimpy prayer. & \\ 
 & & & \\ 
\hline
\end{tabular}
\end{center}
\tablenotetext{1}{The target and caster's {\tt levels} are compared. If the caster's
{\tt level} is higher, then the creature will probably be destroyed.}
\tablenotetext{2}{``holy word'' also defines a class of prayers. These spells are 
all designed to slay only the enemies of the priest's god.}
\end{table}
 
\subsection{Example god}

Lets create an example god$-$the ``god of the undead''. If you worship
the god of the undead, don't expect to be able to gain priest
experience\footnote{i.e. experience for the {\tt wisdom} experience category}
for killing the undead! But you might gain, as a priest of the
undead, greater powers of commanding undead, and experience for
killing certain (living) creatures that serve an enemy god. Each priest
takes on a portion of the ``aura'' of their god; this means that our
priest will probably become protected to life-damaging magic like draining
and death, while conversely becoming more vulnerable to fire. Such a
priest, because their god's domain does not include the living, probably wont be
capable of healing either.
 

\section{Alchemy} \label{sec:alchemy}
\index{alchemy}\index{magic, alchemy}\index{\wizardry\} 
 
Alchemy is a sub-type of \wizardry .  Being an alchemist is easy; 
you only need satisfy the following: 
\begin{quote}
1) be able to cast the \alchemy\ spell.  \\
2) have access to a cauldron\inputimage{cauldron}. \\ 
3) have some ingredients. 
\end{quote}
To create something put ingredients in the cauldron, then
cast the '\alchemy ' \incantation . You might make something :). 
Be warned though! backfire effects are possible, especially if you 
throw a lot of stuff in the pot. In fact, the more junk which is 
in the cauldron, the 
{\em worse} any potential backfire is likely to be. Backfire generally
occurs when you get the ingredients wrong but low-level alchemists
attempting very difficult (4$+$ ingredient) formulae may have 
problems too!
 
In order to get better at making stuff, you will need to learn the 
{\tt alchemy} {\em skill}. Books found in shops (and elsewhere) will give you 
formulae for making stuff.  There is no hard limit on the number of 
formulae which might make something (the code is pretty flexible), 
so you can always {\em experiment} on your own, but this will be dangerous!


\section{Rune magic} \label{sec:rune} 
\index{magic, rune}\index{\wizardry\}\index{runes}

Runes are another special form of \wizardry ; essentially runes are
magical inscriptions on the dungeon floor which cast a spell (or 
``detonate'') when something steps on them.  Flying
objects don't detonate runes.  Beware!  Runes are invisible
most of the time! 
 
  There are several runes which are specialized;  these can be set as
your range spell.  Some of these are:
\vskip 12pt
\begin{quote}
\begin{tabular}{lcl} \index{runes, types} 
  Rune of Fire\inputimage{runefire}    &    -    &         does fire damage \\ 
                   &          &         when it detonates \\
  Rune of Frost\inputimage{runefrost}    &    -    &         does cold damage\\ 
  Rune of Blasting\inputimage{runeblast} &    -    &         does physical damage \\ 
  Rune of Shocking\inputimage{runeshock} &    -    &         does electric damage \\ 
  Rune of Death\inputimage{runedeath}    &    -    &         attacks with attacktype \\
		   &		&	"death" at caster level \\ 
 & & \\ 
 \multicolumn{3}{l}{ + some others you may discover in \wizbook s.} \\  
\end{tabular}
\end{quote}
 
  The spell 'disarm'\index{runes, disarming} may be used to try and destroy a rune you've
discovered.  In addition, there are some special runes which may only 
be called with the 'invoke' command:
\vskip 12pt
\begin{quote}
\begin{tabular}{lcl}
  Magic Rune\inputimage{runegen}     &      -    &         You may store any \incantation\ \\ 
                 &            &         in this rune that you know and \\ 
                 &            &         have the mana to cast. \\ 
 & & \\ 
  Marking Rune\inputimage{runemark}   &      -    &         this is basically a sign.  You \\ 
                 &            &         may store any words you like in \\ 
                 &            &         this rune, and people may apply \\ 
                 &            &         it to read it.  Maybe useful for \\ 
                 &            &         mazes!  This rune will not detonate, \\ 
                 &            &         nor is it ordinarily invisible. \\ 
\end{tabular}
\end{quote}

\subsubsection{Partial Visibility of Runes}

  Your runes will be partially invisible.  That is, they'll be visible
only part of the time.  They have a 1/(your {\tt level}/2) chance of being
visible in any given round, so the higher your level, the better hidden
the runes you make are.

\subsubsection{Examples of usage}\index{runes, usage}
 
Here are several examples of how you can use the runes.\\ 

{\tt 'invoke magic rune heal} 
\begin{quote}
will place a magic rune of healing one square ahead of you, whichever
  way you're facing.
\end{quote}
 
{\tt 'invoke magic rune transfer} 
\begin{quote}
as above, except the rune will contain the spell of transference
\end{quote}
 
{\tt 'invoke magic rune large fireball} 
\begin{quote}
as above, except the spell large fireball will be cast when someone
     steps on the rune.  the fireball will fly in the direction the caster
     was facing when he created the rune.
\end{quote}
 
{\tt 'cast rune of fire} 
\begin{quote}
prepares the rune of fire as the range spell.  Use the direction
      keys to use up your mana and place a rune.
\end{quote}
 
{\tt 'invoke marking rune fubar} 
\begin{quote}
  places a rune of marking, which says "fubar" when someone applies it.
\end{quote}
 
{\tt 'invoke marking rune touch my stuff and I will hunt you down!} 
\begin{quote}
  places the marking rune warning would-be thieves of their danger. 
\end{quote}
 
 
\subsubsection{Restrictions on runes:} 
 
  You may not place runes underneath monsters or other players.  You
may not place a new rune on a square which already has a rune.  Any
attempt to do the latter strengthens the pre-existing rune.



\chapter{Skills System}\index{skills}\index{experience}\label{chap:skills}

\section{Description}\index{skills, description}

Under the skills system the flow of play changes 
dramatically\footnote{The skills system is enabled as the default option 
as of version 
0.92.0}$^{, }$\footnote{The new skills/experience system is compatible 
with character files from at least version 0.91.1 onward.}. 
Instead of gaining experience for basically just killing monsters (and disarming
traps) players will now gain a variety of experience through the use
of skills. Some skills replicate old functions in the game (e.g. melee
weapons skill, missile weapon skill) while others add new functionality
(e.g. stealing, hiding, writing, etc).  A complete list of the available 
skills can be found in table \ref{tab:skill_stats}. Appendix \ref{app:skills} 
contains descriptions for many of the skills. 

\begin{table}
\begin{center}
\caption{Skills \label{tab:skill_stats}}\index{skills, list}\index{skills, associated}
\index{experience, categories}
\index{skills, miscellaneous} 
\small
\vskip 12pt
\begin{tabular}{|clccccc|} \hline 
 & Skill & Experience Category & \multicolumn{3}{c}{Associated Stats} & \\ 
 & & & (Stat 1) & (Stat 2) & (Stat 3) & \\ \hline\hline  
 & & & & & & \\
\input{skill_stat}
 & & & & & & \\
\hline
\end{tabular}
\end{center}
\end{table}


\section{About experience and skills}\index{skills, 
gaining experience}\index{experience}

\subsection{Associated and miscellaneous skills}
\index{skills, associated}\index{skills, miscellaneous}

In \cf\ two types of skills exist; The first kind, ``associated''
skills, are those skills which are {\em associated with a category of 
experience}.  The other kind of skill, ``miscellaneous'' skills,
are {\em not} related to any experience category.

The main difference between these two kinds of skills is in the 
result of their use.
When associated skills are used {\em successfully} experience is 
accrued in the experience category {\em associated with that skill}. 
In contrast, the use of miscellaneous skills {\em never} gains
the player any experience regardless of the success in using it.

{\em Both} miscellaneous and associated skills can {\em fail}. This means
that the attempt to use the skill was unsuccessful. {\em Both} 
miscellaneous and associated skills {\em can} have certain
primary stats {\em associated} with them. These associated stats can help   
to determine if the use of a skill is successful and to what
{\em degree} it is successful. 

All gained experience is modified by the associated 
stats for that skill (table \ref{tab:skill_stats}) and then the 
appropriate experience category automatically updated as needed.

\subsection{Restrictions on skills use and gaining experience}
\index{skills, restrictions}

Neither a character's stats nor the character class restricts the
player from gaining experience in any of the experience 
categories. Also, there are no inherent 
restrictions on character skill use$-$any player may
use any {\em acquired} skill. 

\begin{table}
\begin{center}
\caption{How stats associated with a skill modify gained experience}\label{tab:exp_stat_mod}
\index{skills, stat multipliers}
\footnotesize
\vskip 12pt
\begin{tabular}{|cc|cc|} \hline
Average of  & Experience gained  & Average of & Experience gained \\
associated stats &  multiplier & associated stats &  multiplier \\ \hline\hline
\input{statskmod}
\hline
\end{tabular}
\end{center}
\end{table}

\subsection{Algorithm for Experience Gain under the skills system}

Here we take the view that a player must 'overcome an opponent'
in order to gain experience. Examples include foes killed in combat,
finding/disarming a trap, stealing from some being, identifying 
an object, etc.

Gained experience is based primarily on the difference in levels 
between 'opponents', experience point value of a ``vanquished foe'', 
the values of the associated stats of the skill being used and 
two factors that are set internally\footnote{If you want to 
know more about this, check out the skills\_developers.doc}.

Below the algorithm for experience gain is given where player ``pl'' 
that has ``vanquished'' opponent ``op'' using skill ``sk'':
\begin{quote}
EXP GAIN = (EXP$_{op}$ + EXP$_{sk}$) * lvl\_mult * stat\_mult
\end{quote}
where EXP$_{sk}$ is a constant award based on the skill used, 
EXP$_{op}$ is the base experience award for `op' which depends
on what op is (see below), 
stat\_mult is taken from table \ref{tab:exp_stat_mod}, 
and lvl\_mult is:\\ 

\noindent{For} level$_{pl}$ $<$ level$_{op}$:: 
\begin{quote}
lvl\_mult = FACTOR$_{sk}$ * (level$_{op}$ - level$_{pl}$)
\end{quote}
\noindent{For} level$_{pl}$ $=$ level$_{op}$:: 
\begin{quote}
lvl\_mult = FACTOR$_{sk}$
\end{quote}
\noindent{For} level$_{pl}$ $>$ level$_{op}$:: 
\begin{quote}
lvl\_mult = (level$_{op}/$level$_{pl}$); 
\end{quote}
where level$_{op}$ is the level of `op', level$_{pl}$ is the level
of the player, and FACTOR$_{sk}$ is an internal factor based on
the skill used by pl.

There are three different cases for how EXP$_{op}$ can be computed:
\begin{quote}
1) {\bf op is a living creature}: EXP$_{op}$ is just the base 
experience award given in the \spoiler . \\

2) {\bf op is a trap}: EXP$_{op} \propto$ 1/(the time for which the
trap is visible). Thus, traps which are highly {\em visible} get {\em lower}
values. \\

3) {\bf op is not a trap but is non-living}: EXP$_{op}$ = internal
experience award of the item. Also, the lvl\_mult is multiplied by
any {\tt magic} enchantment on the item.
\end{quote}

\section{How skills are used}\index{skills, how to use}
 
\begin{table}
\small
\caption{Skills commands}\label{tab:skill_cmd}
\vskip 12pt
\begin{center}
\begin{tabular}{|cllc|} \hline 
 & & & \\
 & {\tt skills} & 	This command lists all the player's & \\ 
 &		& current known skills, their level & \\ 
 &		& of use and the associated experience & \\ 
 &		& category of each skill. & \\ 
 & & & \\ 
 & {\tt ready\_skill $<$skill$>$} 	& This command changes the player's & \\ 
  &				& current readied skill to {\tt $<$skill$>$}. &  \\ 
 & & & \\
 & {\tt use\_skill $<$skill$>$ $<$string$>$}  & This command changes the player's & \\ 
 &				& current readied skill {\em and} then & \\ 
  & 				& executes it in the facing direction & \\ 
 &				& of the player. Similar in action to & \\ 
 &				& the {\tt invoke} command. & \\ 
 & & & \\ \hline 
\end{tabular}
\end{center}
\end{table}
 
Three player commands are related to skills use: {\tt ready\_skill}, 
{\tt use\_skill}, and {\tt skills} (see table \ref{tab:skill_cmd}). 
Generally, a player will use a skill by first readying the right one,
with the {\tt ready\_skill} command and then making a ranged ``attack'' to
activate the skill; using most skills is just like firing a wand or a
bow.  In a few cases however, a skill is be used just by having it
{\em readied}. For example, the {\tt mountaineer} skill allows
favorable movement though hilly terrain while it is readied.
 
To change to a new skill, a player can use either the
{\tt use\_skill} or {\tt ready\_skill} commands, but note that the use of
several common items can automatically change the player's current
skill too. Examples of this include readying a bow (which will cause the
code to make the player's current skill {\tt missile\_weapons}) or readying
a melee weapon (current skill auto-matically becomes {\tt melee weapons}).
Also, some player actions can cause a change in the current skill.
Running into a monster while you have a readied weapon in your inventory
causes the code to automatically make our current skill {\tt melee weapons}.
As another example of this$-$casting a spell will cause the code to
switch the current skill to {\tt \spellcasting} or {\tt praying} (as appropriate
to the spell type).
 
It is not possible to use more than one skill at a time.
 
\section{Acquiring skills}\index{skills, learning}\index{skills, tools}

Skills may be gained in two ways. In the first, new skills may {\em learned}.
This is done by reading a ``skill scroll'' and the process is very similar
to learning a spell. Just as in attempts to learn \incantation s, success in 
learning skills is dependent on a random test based on the learner's INT.
Using your INT stat, look in the learn\% column in table \ref{tab:pri_eff} 
to find your \% chance of learning a skill. Once you hit 100\% you will 
always be successfull in learning new skills. 

The acquisition of a {\em skill tool} will also allow the player to use
a new skill. An example of a skill tool is ``lockpicks''\inputimage{lockpicks}
(which allow the
player to pick door locks). The player merely applies the skill
tool in order to gain use of the new skill. If the tool is unapplied,
the player looses the use of the skill associated with the tool.

After a new skill is gained (either learned or if player has an applied
skill tool) it will appear on the player's skill roster (use the
'skills' command to view its status). If the new skill is an associated
skill, then it will automatically be gained at the player's current level 
in the appropriate experience category. For example, Stilco the Wraith, 
who is 5th level in {\tt agility}, buys a set of lockpicks and applies them.
He may now use the skill lockpicking at 5th level of ability since that 
is an {\tt agility} associated skill.


\chapter{Equipment}\label{chap:equip}

\section{Going to market..}\index{equipment, value}
\inputimage{shops}

You can find equipment for sale at easily recognizable buildings. To buy an
item just pick it up and walk out of the building by stepping on a 
shop mat\inputimage{shopmat}. The cost of the item will auto-matically be
deducted from your money\inputimage{money}. To sell an item, 
enter the shop and drop the item
on the shop floor. Money from the sale will auto-matically be placed in your
inventory. Use the {\tt examine} command, or the cursor and left button
of the mouse to examine the price of an item {\em before} you buy or sell. 
 
\subsection{Some notes about shopping}
\index{shopping}
 
Most items will have a value based on their ``standard'' cost
multiplied by a factor based on your charisma (see table \ref{tab:pri_eff}). 
You can never look good enough that you can buy stuff then sell
it at a profit. \\ 

\noindent{Some} notable exceptions to the above:
\begin{quote} 
   $\bullet$ Gems\index{gems}\inputimage{gems} will always be sold or bought 3\% more or less their standard value. \\ 
   $\bullet$ For magic stuff value is 3$\times$({\tt magic})$^{3}$ of standard value. \\ 
   $\bullet$ Unidentified items value is 2/3 of standard. 
\end{quote}
 
\subsection{Plundering shops}
\index{stealing from shops}

It is not possible to steal from shops (sorry!). If you somehow make
it out of a shop with ``unpaid'' items, you will find that they will
be unusable until paid for. On another note, if you save yourself with
unpaid items in a shop, then crash the game and reload, you will 
find that the unpaid items will not be saved.
 
\section{Items}
\label{sec:items}

In this section we detail some interesting properties of various 
bits of equipment which may be found in \cf . \\ 

\indent{\bf Books:}\inputimage{books} \\ 
\index{books}
% \index{spells, how to learn}
This is how players can obtain magical spells, sometimes a player can
learn the spell, other times they cannot. The chance depends on the type
of spell, either INT (\incantation s) or WIS (prayers) is used to help
determine the percentage chance that the spell might be learned (see section
\ref{sec:spell_learn} for details).

There are many, many different types of books out there,
as well as being spell books (\wizbook s and prayerbooks), the 
following information can 
appear in books generated in shops and/or monster treasure hoards$-$ 
\begin{quote}
        $\bullet$ Compendiums on monsters. Their powers/abilities are
            described as in the \spoiler . 

        $\bullet$ Compendiums of \incantation s/prayers by spell Path. Higher
            level texts are more complete in their description of
            available spells.

        $\bullet$ ``Bibles'': various aspects, properties, and characteristics
            of a God/cult are described. Higher level texts
            have more/better information. 

        $\bullet$ Compendiums explaining the powers of magic items. Higher
            level texts have more items detailed. 

        $\bullet$ Alchemical Formulae. 

        $\bullet$ Other randomly generated information.
\end{quote}
Book level is assigned when the book is generated as treasure.
Level is based on the difficulty of the map the book is
generated on. All information is {\em server} specific. \\ 

{\bf Scrolls and Potions:}\inputimage{potions}\inputimage{scrolls}
\index{scrolls}\index{potions}\\ 

	Most of these items provide a one-shot use of a spell without
	making the user expend either mana or grace. Scroll use
	depends on the user's {\tt literacy} skill and may fail. Potions 
	always work, but are more expensive to buy. Several kinds
	of items are classed as "potions": balms, figurines, and
	dusts. Some potions don't cast spells, but instead raise
	the drinker's stats. Beware cursed potions. They can {\em lower}
	your stats instead! \\ 

{\bf Wands(Staves)/Rods/Horns:}\inputimage{wands}\inputimage{rods}
\inputimage{horns}\index{wands}\index{rods}\index{horns} \\ 
	These items provide use of spells. Wands have a limited
	number of charges, while horns and rods will recharge 
	(but don't fire as much damage in a small amount of time).
	Horns are used at the overall level of ability of the 
	user, while rods and wands cast spells at the item level.\\ 

{\bf Rings:}\inputimage{rings} \\ \index{rings} 
     Many different types, rings can be worn to add/remove different
     immunities, gain/lose spell Paths and alter all types of stats. \\ 
 
{\bf Food/Flesh:}\inputimage{food}\inputimage{flesh}\index{food}\index{flesh} \\  
	These items provide sustenance. Food is generally more healthy
	to eat, while some flesh items can be sold for good cash. Both
	types may temporarily alter your stats, and/or be poisonous.
	Many flesh items inherit the properties of the monster they
	came from. For example, a ``poisonous'' monster will leave 
	behind poisonous flesh. Don't eat it if you know what's good
	for you!!\\ 

{\bf Weapons/Armour: }\index{armour}\index{weapons}\\ 
     Tons of items, it is up to you as the player to figure out which work
     better then others. Take a look at weapon/armour weight in the \spoiler\ 
	to get an idea of how enchanted unidentified items are. \\ 
 
{\bf Artifacts:}\index{artifacts}\\ 
     These are the real treasures of the game. There are more than 20
     artifacts out there, but they are very hard to come by. \\ 

\section{Encumbrance}\label{sec:encumberance}\index{equipment, encumbrance}\index{encumbrance}
Armour, weapons, shields will encumber a wizard and cause spell 
failure.  Light equipment causes no failure at all whereas heavy equipment 
causes mondo failures.

The reasoning is that the bulkiness of objects, not their weight exactly, is 
what causes failures. So the basic idea of encumbrance is that items get in the 
way more than they weigh down. Unfortunately, our only measure of 'getting 
in the way' was the weight.
 
\subsection{How encumbrance is calculated}
Encumbrance points are tallied only from {\em applied} objects. Weapons 
give
3x their weight in kg in encumbrance points. Shields give 1/2 their weight
in kg in encumbrance points. Armour gives its weight in encumbrance points.
 
There's an allowance of encumbrance points which all players get before they
start losing \incantation s, this was about 35-45, not too much.
 
The formula works like this: You make a roll of 1-200. You compare it to a
failure threshold. This threshold is: encumbrance + \incantation\ {\tt level}
 - caster {\tt level} - 35
 
For example, lets say a 4th {\tt level} wizard is casting a 5th {\tt level} 
\incantation . The wizard is wearing plate mail (100 kg), a 20 kg shield and 
wielding a 15 kg weapon. His encumbrance is 100 + 10 + 45 = 155. Thus, his
threshold for failure is 155 + 5 - 4 = 156 or just about 3/4 failure rate.
 
There is no special bonuses for using magical equipment, although, it is
clear that magical armour and weapons make things better through their weight. 

\section{Enchantments}\label{sec:enchant}
\index{equipment, magic}
 
Some items will have numerical values such as +1, +2, +3, etc.
trailing their names. These {\em magic} values indicate that the item 
is enchanted,
and in some way may be better or (if the value is negative) worse
than ordinary runaday items of its kind.
 
\subsection{Enchanting armour}\index{enchantment, armour}\index{armour} 

Enchantment of armour is achieved with the 
{\em enchant armour}\inputimage{scrolls}\ scrolls. 
Each time you successfully use a scroll, you will
add a plus value, more armour to the piece of equipment and
some fractional amount of weight.

You may only add up to 1 + (overall {\tt level}/10) (rounded down like an
integer) in pluses to any one piece of armour. How much 
armour value you add to the item is also dependent on your
overall {\tt level}. You may never enchant a piece of armour to 
have an armour rating greater than their overall {\tt level} or 99.
  
\subsection{Enchanting weapons}\index{enchantment, weapons}
\index{weapons}

This is done via a series of scrolls\inputimage{scrolls} that you 
may find or buy in
shops. The procedure is done in two steps. Use the {\em prepare weapon} scroll
to lay a magic matrix on your weapon. Then use any of the other
scrolls to add enchantments you want. Note that some of these scrolls will
also require a ``sacrifice'' to be made when they are read. To sacrifice
an object just stand over it when you read the weapon scroll. Scrolls 
for weapon enchantment are: \\

{\bf Prepare weapon} \\
Diamonds are required for the sacrifice. The item
can be enchanted the square root of the number of diamonds sacrificed. Thus,
if 100 diamonds are sacrificed, the weapon can have 10 other enchant scrolls
read. \\ 

{\bf Improve damage} \\ 
     There is no sacrifice. Each scroll read will increase the damage by 5
     points, and the weight by 5 kilograms. \\ 

{\bf Lower (Improve) Weight} \\ 
     There is no sacrifice. Each scroll read will reduce the weight by 20\%.
     The minimum weight a weapon can have is 1 gram. \\ 

{\bf Enchant weapon} \\ 
     This does not require any sacrifices, and increases the magic by 1. \\ 

{\bf Improve Stat} (ie, Strength, Dexterity, etc) \\ 
     The sacrifice is the potion\inputimage{potion} of the same type 
as the ability to be
     increased (ie, Improve Strength requires strength potions). The number
     of potions needed is the sum of all the abilities the weapon presently
     gives multiplied by 2. The ability will then be increased by 1 point.
     Thus, if a sword is Int +2 and Str +2, then 8 potions would be needed to
     raise any stat by one point. But if the sword was Int +2, Str +2, and Wis -2,
     then only 4 potions. A minimum of 2 potions will be needed. \\ 

{\bf WARNING:} something to keep in mind before you start enchanting 
like crazy$-$you can only use a weapon that has 5 + 1 enchantments 
for every 5 levels of 
{\tt physique} experience you possess. So, a character with 
10th level in the {\tt physique} experience category may only be able to 
use a weapon with a maximum of 7 enchantments!


\chapter{Hints on playing \cf}

This section amounts to spoilers for the game. If you don't like
that kind of information, don't read any further in this chapter!

\section{Beginning players}\label{sec:first}\index{beginning players}

\noindent{I'm on the starting map, what do I do now?}
You should be in a city square with a few sign posts in the middle.
Move over the signs and {\tt apply} them ("{\tt A}") to read what they say.

For beginners, there are several maps designed for them. Find these areas
and clear them out. All throughout these levels, a player can find signs and
books which they can read by stepping onto them and hitting '{\tt A}' to apply the
book/sign. These messages will help the player to learn the system.
Probably the first, best area for a beginning player to start out in
is ``Beginners''. This area is a small house located south-west of your
starting location.

\subsubsection{Flailing about with spells}
Some items are perishable. If you shoot a
fireball into a room full of scrolls, you will notice them
going up in smoke! So be careful not to destroy valuable items.

\subsubsection{Tips on surviving}
\cf\ is populated with a wealth of different monsters. These monsters
can have varying immunities and attacktypes. In addition, some of them can
be quite a bit smarter than others. It will be important for new players to
learn the abilities of different monsters and learn just how much it will
take to kill them.

Most monsters in the game are out to mindlessly kill and destroy the
players. Killing monsters will help boost a player's score.
When fighting a large amount of monsters in a single room, attempt to
find a narrower hallway so that you are not being attacked from all sides.
Charging into a room full of Beholders would not be wise,
instead, open the door and fight them one at a time.

\section{Priorities for low-level characters}\label{sec:low}

The priority for characters below about 5th level is to gain some
basic items. In this regard, better armour and better spells are best.
Look for a
quest among the various islands that will allow you to obtain
mithril mail. Whenever you scrape together $\sim$100$-$200 platinum
pieces go shopping for armour and weapons (or spells). If you are a
fighter type, try to have at least a +2
weapon, +2 helmet, +2 suit of armour, and +2 shield before you reach 5th level.
For wizards, attempt to recover enough treasure to be able to buy
up good attack spells. For priests, first thing to do is worship a god!
Try to get the {\em holy word} prayer as soon as possible. Make sure
your current god allows good potential use of this spell.

For all classes, get access to the {\em detect magic} \incantation\
as soon as possible.
This will allow you to sort through the treasure you find while you're in
the dungeon, and will save you time and money at the shops.


% APPENDICES
\appendix
\chapter{Player Commands}
\label{app:commands}
{\scriptsize
\begin{longtable}{p{4cm}p{9cm}}
apply $<$string$>$ & Without any parameter, applies the top object in the ground view. With a string, applies the first object matching the string. \\ 
bind & Used to bind commands to keys. \\ 
brace & Toggles brace status. While braced you will not move. \\ 
cast $<$spell$>$ & Readies named spell. \\ 
clearinfo & Clears the text screen. \\ 
disarm & Executes a disarming action in facing direction. \\ 
dm $<$password$>$ & Player becomes the dm. You character is unsavable after this. \\ 
drop $<$object$>$ & Drops object from player inventory. All items matching $<$object$>$ are dropped, unless they are locked or cursed. \\ 
dropall &  Drops all but locked objects in inventory. \\ 
east & Execute a move east. \\ 
examine $<$object$>$ & Examines object, defaults to top object in look window. \\ 
get & Picks up top object in look window. \\ 
gsay & Sends a message only to members of your party.\\ 
help $<$subject$>$ & Help on subject, defaults to shows list of available help. If $<$subject$>$ is {\tt commands}, will display a list of available commands. \\
hiscore & Show list of highest player scores. \\ 
inventory & Lists all objects in your inventory in text window. \\ 
invoke $<$spell$>$ $<$string$>$ & Readies, then casts a spell. String is optional \\
listen & Changes the level of your hearing. \\ 
mapinfo & Show information about current map. \\ 
maps $<$name$>$ & Show status of all active maps. If $<$name$>$ is provided, only lists maps with matching name. \\
motd & Show message of the day again. \\ 
north & Execute a move north. \\ 
northeast & Ditto to north east. \\  
northwest & Ditto to the north west. \\ 
output-count & Toggles message grouping, so that identical messages are put together. \\ 
output-sync &  No idea what this does. \\
party $<$command$>$ & Set of commands used with simple party system. \\ 
peaceful & Toggle peaceful status. \\ 
pickup $<$command$>$ & Change pickup status by number or value density. \\ 
prepare & Alias for cast command. \\ 
quit & Quit the game and totally removes your character. \\ 
ready\_skill $<$skill$>$ & Prepare a skill for use. \\ 
rotateshoottype & Rotate the range slot by 1. \\ 
save & Save your current position and status. \\ 
savewinpos & Save your current window layout. \\ 
say & Say something to all players on your map. \\ 
search & Execute a search in all nearby squares.\\ 
search-items $<$command$>$ & Change the status of items searched for. \\ 
shout & Send a message to all players regardless of map.\\ 
show & \\ 
showinvicon & Toggle status of invicon in inventory window. \\ 
skills & Show all available skills, experience categories and level. If a string is provided, only lists matching skills.\\ 
south & Execute a move south.\\ 
southeast & Ditto to southeast. \\ 
southwest & Ditto to southwest. \\ 
stay & \\ 
take & Alias for pickup. \\ 
tell $<$who$>$ $<$msg$>$ & Tell character who the msg. \\ 
throw & Throws an item. \\ 
time & Displays the ingame time \\ 
title $<$string$>$ & Change your title to string.\\ 
unbind $<$command$>$ & A set of commands to reverse key bindings. \\ 
use\_skill $<$skill$>$ $<$string$>$ & Ready, then use a skill. \\
version & Print out all the contributors to \cf .\\ 
west & Execute a move west. \\ 
who & Show what players are currently logged on. If a parameter is specified, only displays players in regions with matching name. \\
wimpy $<$percent$>$ & Auto-matically run away when hp$<$\%$\times$Max hp. \\ 
\end{longtable}

Commands for Dungeon Masters only:
\begin{longtable}{p{4cm}p{9cm}}
malloc & Shows which stuff is taking up how much memory. \\ 
sstable & Sends to the server log statistics about shared strings. \\ 
strings & Shows the status of shared sting parameters. \\ 
sync $<$integer$>$ & Change sync value to integer. \\ 
\end{longtable}
}

\chapter{Skills}
\label{app:skills}
\index{skills, description}

The following is the current (7/15/96) roster of skills and the
description for each. Use the command {\tt crossfire -m5} to see the array
of skills and experience in your version of crossfire.
(Note: you need to have compiled with the {\tt DUMP\_SWITCHES} and
{\tt ALLOW\_SKILLS} flags for this to work!)
 
Emphasis on type denotes a skill which monsters/NPC's may also use.
 
\begin{longtable}{|p{4cm}|p{9cm}|} \hline
alchemy &  User can identify potions, containers, flesh parts, \\ 
	&  and amulets. \\ 
 &  \\
bargaining   	& While this skill is readied the user has added CHA \\ 
	 	& for purposes of purchase and selling of items only. \\ 
		& Cha is never allowed to exceed 30. \\ 
 &  \\
bowyer  &  User can identify missile weapons and missiles. \\ 
 &  \\
{\em clawing}	& User can make a {\em bare-handed attack}. Damage \\ 
                & is based on the user's Str and {\tt level}. This is the \\ 
		& default ``hand-to-hand'' fighting skill for the  \\
		& Quetzecoatl character class. \\
 &  \\
find traps	& User can search (more effectively) for traps. \\ 
		& Not a 'passive' skill, it is applied in order \\ 
		& to gain the advantage in discovering traps. \\ 
 &  \\
{\em flame touch} & User can make a {\em bare-handed attack}. Damage \\ 
		& is based on the user's Str and {\tt level}. This \\ 
		& is the default hand-to-hand fighting skill \\ 
		& for the Fireborn character class. \\ 
 &  \\
{\em hide}	& User enjoys limited form of invisibility. If \\ 
		& they attack or move too much they become visible. \\ 
 & \\
jeweler		& User can ident gems and rings that they hold. \\ 
 & \\
{\em jumping}	& User can 'skip' over 1-2 spaces in a selected \\ 
		& direction. Distance depends on weight carried, \\ 
		& STR and DEX of the user. This skill may also \\ 
		& be used as an attack. \\ 
 & \\
{\em karate}	& User can make a {\em bare-handed attack}. Damage \\ 
		& is based on the user's Str and {\tt level}. This attack \\ 
		& is the fastest and (at higher levels) most deadly \\ 
		& of the hand-to-hand attacks available. \\ 
 & \\
literacy	& User can ident books and scrolls that they hold. \\ 
		& It also allows the player to read scrolls, books, \\ 
		& \wizbook s and praybooks. \\ 
 & \\
{\em lockpicking} & User may 'pick locks' (open doors). User needs \\ 
		& to have readied 'lockpicks' to use this skill. \\ 
 & \\
meditation	& Player can regain mana/hp at an accelerated rate. \\ 
		& Player must first strip off encumbering armour \\ 
		& however. This skill is only available to the Monk \\ 
		& character class. \\ 
 & \\ 
melee weapons	& User may use hand-held weapons (eg swords, \\ 
		& spears, mace, etc). \\
 & \\ 
missile weapons	& The user is allowed to make attacks with \\ 
		& ranged weapons (eg bow, crossbow). \\ 
 & \\
mountaineer	& While the skill is readied, the possessor will \\ 
		& move faster through "hilly" terrain (hills, \\ 
		& mountains, etc.) \\ 
 & \\
oratory		& User may 'recruit' followers. Recruitees must be \\ 
		& of lower {\tt level}, and unaggressive to start. Use \\ 
		& of this skill may anger the audience. Also, \\ 
		& 'special' monsters are immune to recruitment. \\ 
		& Success depends on User Cha and {\tt level}. \\ 
 & \\
praying		& User is allowed to cast ``priest'' spells. In addition, \\ 
		& this skill may be used to accelerate the accumulation \\ 
		& of grace. This skill may be either {\em learned} or \\ 
		& acquired through the use of a {\em holy symbol}. \\ 
 & \\
{\em punching}	& User can make a {\em bare-handed attack}. Damage \\ 
		& is based on the user's Str and {\tt level}. This is \\ 
		& the most feeble of the hand-to-hand attacks. \\ 
 & \\
remove traps 	& User may remove previously discovered traps. \\ 
 & \\
sense curse	& User may discover whether items that they hold \\ 
		& are {\em cursed}. \\
 & \\ 
sense magic	& User may discover whether items that they hold \\ 
		& are {\em magic}. \\
 & \\ 
set traps	& Unimplemented. \\ 
 & \\ 
singing		& User may pacify hostile monsters with this skill. \\ 
		& Certain kinds of monsters are immune. Success \\ 
		& depends on user {\tt level} and CHA. \\
 & \\ 
smithery	& User may ident arms and armour that they hold. \\ 
 & \\ 
\spellcasting\ 	& User is allowed to cast ``wizard'' spells. This \\ 
		& skill may be acquired either through the use \\ 
		& of a {\em talisman} or {\em learned} via a skill scroll. \\ 
 & \\ 
{\em stealing}	& User can steal items from other creatures. \\ 
 & \\
thaumagragist	& User can ident rods, wands and horns that they \\ 
		& are holding. \\ 
 & \\
throwing	& Unimplemented. \\ 
 & \\
use magic item	& User can use magic items like rods/wands/horns. \\ 
 & \\
woodsman	& While the skill is readied, the possessor will \\ 
		& move faster through ``wooded'' terrain (forest, \\ 
		& grasslands, brush, jungle, etc.). If actively used, \\
		& this skill will identify foods and flesh parts. \\ 
 & \\
writing		& User may both write messages in books {\em and} rewrite \\ 
		& spell scrolls with a previously known spell. For \\ 
		& new spell scrolls mana, time and an old scroll are \\ 
		& needed. Backfire effects are possible. To write \\
		& text in a book, use the skill as: {\tt use\_skill} \\
		& {\tt writing $<$msg$>$}. This skills is only available as a \\
		& 'writing pen'; the literacy skill must be possessed \\
		& before this can be used. \\ \hline
\end{longtable}

\chapter{Description of Gods}
\label{app:gods}
\index{gods, description}

Below in boxes, the gods in your compiled version of \cf\ are shown.
Use the command {\tt crossfire -m8} to check if the information
presented here is accurate.
(Note: you need to have compiled with the {\tt DUMP\_SWITCHES} and
{\tt MULTIPLE\_GODS} flags for this to work!) The boxed attributes
have meaning as follows:
\vskip 12pt
\begin{tabular}{ll}
Enemy cult: & Name of the enemy god \\
Aligned race(s): & Names of races friendly to the cult. The priest \\
	 & of this cult has greater power over these creatures. \\
	 & In some cases the prayer {\tt summon cult monsters} \\
	 & will summon these monsters to help the priest. \\
Enemy race(s): & Names of races hated by the cult. ``Holy word'' \\
	 & prayers of this god may be used to kill these \\
	 & creatures. \\
Attacktype(s): & Attacktypes used by this god's avatar and \\
	 & by cult {\tt cause wounds} prayers. \\
Immunity: & Granted by the {\tt holy possession} prayer. \\
Protected: & Granted to a cult priest and by the {\tt bless} \\
	 & prayer. \\
Vulnerable: & Given to a cult priest and by the {\tt curse } \\
	 & prayer. \\
Attunded: & The cult priest is attuned to these spellpaths. \\
Repelled: & The cult priest is repelled to these spellpaths. \\
Denied: & The cult priest is denied use of these spellpaths. \\
Added gifts/limits: & A cult priest has these addtional benefits\\
 	& and restrictions. \\
\end{tabular}
\vskip 12pt
Note that not all gods have values for all possible attributes. In
this case, no attribute will appear in that god's box.

\begin{longtable}{|p{4cm}p{9cm}|} \hline
\input{gods.tex}
\end{longtable}



% Index
\input{index}

% REFERENCES
% \input{references}

\end{document}		   % You are done !

